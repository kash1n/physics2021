\section{Идеальный классический газ. Распределение Больцмана.}
\textbf{Идеальный газ} - газ, взаимодействие между частицами которого настолько мало, что им можно пренебречь (то есть либо частицы слабо взаимодействуют, либо газ разреженный).
\begin{itemize}
	\item $\varepsilon_k$ - уровень энергии отдельной молекулы ($E_n = \sum_{k}\varepsilon_k$)
	\item $n_k$ - число частиц в состоянии k (число заполнения)
\end{itemize}
Для достаточно разреженного газа $\overline{n_k} << 1$. Это позволяет пренебречь квантовомеханическим взаимным влиянием и применить формулы распределения Гиббса.\\
Тогда вероятность молекуле находиться в k-том состоянии с среднее число молекул в этом состоянии пропорциональны $e^{-\varepsilon_k/T}$:
$$
\overline{n_k} = ae^{-\frac{\varepsilon_k}{T}}
$$
где a - постоянная, определяющаяся условием нормировки $\sum_{k}\overline{n_k} = N$.\\
Распределение молекул идеального газа по различным состояниям, определяемое этой формулой, называется \textbf{распределением Больцмана}.
\subsection{Eщё один способ вывести распределение Больцмана}
Применим распределения Гиббса к совокупности всех частиц газа, находящихся в данном квантовом состоянии.\\
В общей формуле распределения Гиббса
$$
\omega_{nN} = exp (\frac{\Omega + \mu N - E_{nN}}{T})
$$
полагая
$$
N = n_k
$$
$$
E = n_k \varepsilon_k
$$
и приписывая индекс величине $\Omega$, получим распределение вероятностей различных значений:
$$
\omega_{n_k} = exp (\frac{\Omega_k + n_k(\mu - \varepsilon_k)}{T})
$$
В частности $\omega_0 = exp (\Omega_k /T)$ есть вероятность полного отсутствия частиц в данном состоянии. В интересующем нас здесь случае , когда $\overline{n_k} << 1$, вероятность $\omega_0$ близка к 1, поэтому в выражении для $\omega_1$ для вероятности наличия одной частицы в k-том состоянии можно положить, опуская члены высшего порядка малости $exp (\Omega_k /T) = 1$. Тогда 
$$
\omega_1 = exp (\frac{\mu - \varepsilon_k} {T})
$$
А для вероятностей значений $n_k > 1$ - они в том же приближении должны быть положены равными нулю. Поэтому
$$
\overline{n_k} = \sum_{n_k} \omega_{n_k}n_k = \omega_1 \cdot 1
$$
и мы получаем \textbf{распределение Больцмана} в виде
$$
\overline{n_k} = exp (\frac{\mu - \varepsilon_k}{T})
$$
\subsection{И ещё один способ вывести распределение Больцмана (можно не писать в билет я думаю)}
Всякое макроскопическое состояние идеального газа можно характеризовать следующим образом: распределим все квантовые состояния отдельной частицы газа по группам, каждая из которых содержит близкие состояния (обладающие, в частности, близкими энергиями), причем как число состояний в каждой группе, так и число находящихся в них частиц все же очень велики.\\
Перенумеруем эти группы состояний номерами j = 1,2,... и пусть $G_i$ - число состояний в j-той группе, $N_j$- число частиц в этих состояниях. Тогда набор $N_j$ полностью характеризует макроскопическое состояние газа.\\
Задача о вычислении энтропии газа сводится к задаче об определении статистического веса:
$$
\Delta \Gamma = \prod_{j}\Delta \Gamma_j
$$
где $\Delta \Gamma_j$ - статистический вес группы из $N_j$ частиц.
$$
\Delta \Gamma_j = \frac{G_j^{N_j}}{N_j!}
$$
\begin{itemize}
	\item $G_j^{N_j}$ - число вариантов поместить каждую из $N_j$ частиц в одно из $G_j$ состояний (но среди них есть тождественные, отличающиеся перестановкой частиц)
	\item $N_j!$ - число перестановок $N_j$ частиц
\end{itemize}
Энтропия газа:
$$
S = \ln \Delta \Gamma = \sum\ln \Delta \Gamma_j = \sum_{j}(N_j\ln G_j- \ln N_j!) \approx 
$$
$$
\approx \sum_{j}(N_j\ln G_j- N_j \ln\frac{N_j}{e}) = \sum_{j}N_j\ln\frac{eG_j}{N_j}
$$
Введем средние числа $\overline{n_j}$ частиц в каждом из квантовых состояний j-той группы:
$$
\overline{n_j} = \frac{N_j}{G_j}
$$
Тогда
$$
S = \sum_{j}G_j\overline{n_j} \ln \frac{e}{\overline{n_j}}
$$
Если движение частиц квазиклассично, то в этой формуле можно перейти к распределению частиц по фазовому пространству. Разделим фазовое пространство на участки $\Delta p^{(j)}\Delta q^{(j)}$, каждый из которых мал, но содержит все же большое число частиц.
$$
G_j= \frac{\Delta p^{(j)}\Delta q^{(j)}}{(2\pi\hbar)^r} = \Delta \tau ^{(j)}
$$
где r -число степеней свободы частицы, а числа частиц в этих состояниях напишем в виде 
$$
N_j	= n(p,q)\Delta \tau ^{(j)}
$$
где $n(p,q)$ - плотность распределения частиц в фазовом пространстве.\\
Имея в виду, что участки $\Delta \tau ^{(j)}$ малы заменяем суммирование по j интегрированием по всему фазовому пространству частицы:
$$
S = \int n \ln \frac{e}{n}d\tau
$$
В состоянии равновесия энтропия должна иметь максимальное значение. Из этого требования можно найти функцию распределения частиц газа  в состоянии статистического равновесия. Задача заключается в поиске таких $\overline{n_j}$, при которых сумма 
$$
S = \sum_{j}G_j\overline{n_j} \ln \frac{e}{\overline{n_j}}
$$
имеет максимальное значение, возможное при дополнительных условиях, выражающих постоянство полного числа частиц и полной энергии газа:
$$
\sum_{j}N_j = \sum_{j}G_j\overline{n_j} = N
$$
$$
\sum_{j}\varepsilon_jN_j = \sum_{j}\varepsilon_jG_j\overline{n_j} = E
$$
Следуя методу неопределенных множителей Лагранжа надо приравнять к нулю производные:
$$
\frac{\partial}{\partial\overline{n_j}} (S + \alpha N + \beta E) = 0
$$
$$
G_j(-\ln\overline{n_j} + \alpha + \beta \varepsilon_j) = 0
$$
Откуда получим 
$$
\ln\overline{n_j} = \alpha + \beta \varepsilon_j
$$
$$
\overline{n_j} = e ^{\alpha + \beta \varepsilon_j}
$$
Это - ни что иное, как распределение Больцмана. При чем $\alpha$ и $\beta$ связаны с $T$ и $\mu$ посредством $\alpha = \mu / T$, $\beta =  - 1 / T$
