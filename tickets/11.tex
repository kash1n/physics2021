\section {Движение в центрально-симметричном поле}
Рассмотрим движение частицы в стационарном поле 
$$U = U(r), r = |r| = \sqrt{(x^2 + y^2 + z^2)}$$

Такое поле называется {\bf центрально-симметричным}, или, коротко,
{\bf центральным}. В этом случае гамильтониан

$$\hat H = \frac{\hat p^2}{2m_e} + U(r)$$

коммутирует с оператором орбитального момента:

$$[\hat H, \hat L] = 0$$

Для начала покажем, что 
$$\hat p^2 = \hbar^2 \nabla = - \frac{\hbar ^2}{r^2} \frac{\partial}{\partial r} \left( r^2 \frac{\partial}{\partial r} \right) + \frac{\hat L^2}{r^2}$$

Иммеем:
$$\hat L^2 = (r \times \hat p)^2 = - (\hat p \times r) \cdot (r \times \hat p) = - \hat p \cdot (r \times (r \times \hat p)) = - \hat p \cdot (r (r \cdot \hat p) - r^2 \hat p)$$

Из соотножения $[\hat p_n, x_m] = - i \hbar \delta_{nm}$ следует
$$[\hat p \cdot r, r \cdot \hat p] = - 3 i \hbar,$$
$$[\hat p, r^2] = - 2 i \hbar r,$$

используя которые, находим
$$\hat L^2 = r^2 \hat p^2 - (r \cdot \hat p)^2 + i \hbar r \cdot \hat p.$$

Далее:

$$r \cdot \hat p = -i \hbar r \cdot \nabla = -i \hbar r \frac{\partial}{\partial r}$$

$$(r \cdot \hat p)^2 = - \hbar^2 \left( r \frac{\partial}{\partial r} + r^2\frac{\partial^2}{\partial r^2} \right).$$

В итоге получаем выражения для $\hat p^2$, откуда сразу следует $[\hat L, \hat p^2] = 0$ и, следовательно, $[\hat H, \hat L] = 0$.

Стационарное уравнение Шрёдингера для частицы в центральном поле принимает вид:

$$ \left[ \frac{\hbar ^ 2}{2 m_e r^2} \frac{\partial}{\partial r} \left (r^2 \frac{\partial}{\partial r}
\right) + \frac{\hat L^2}{2 m_e r^2} + U(r) \right] \psi = E \psi$$

Так как орбитальный момент $\hat L$ - интеграл движения (из того, что коммутирует с гамильтонианом), то любая собственная функция гамильтониана представляется в виде радиаольной функции, зависящей только от $r$ и сферической функции от $\theta, \phi$

$$\psi (r, \theta, \phi) = R(r) Y_l^m (\theta, \phi)$$.

Очевидно, что на $ Y_l^m (\theta, \phi)$ в уравнении Шредингера можно сократить. Для радиальной функции получаем уравнение

$$ \left[ \frac{\hbar ^ 2}{2 m_e r^2} \frac{\partial}{\partial r} \left (r^2 \frac{\partial}{\partial r}
\right) + \frac{\hbar ^ 2 l (l + 1)}{2 m_e r^2} + U(r) \right] R(r) = E R(r)$$

Удобно ввести новую функцию

$$R(r) = \frac{\chi (r)}{r}$$

Она удовлетворяет {\bf Радиальному уравнению Шрёдингера}:

$$ \left[ \frac{\hbar ^ 2}{2 m_e r^2} \frac{d^2}{d r^2} + U_l(r) \right] \chi = E \chi,$$

где введен {\bf эффективный потенциал}

$$U_l(r) = U(r) + \frac{\hbar^2 l (l + 1)}{2 m_e r^2}$$

Условие нормировки $\chi (r)$ совпадает с условием нормировки одномерной волновой функции:

$$\int d^3 x |\psi|^2 = \int_0^\infty drr^2|R|^2 = \int_0^\infty dr |\chi|^2 = 1$$

Выясним ассимптотику радиальной функции $\chi (r)$, при $r \rightarrow 0$.
Оставляя наиболее сингуларные члены в радиальном уравнении Шрёдингера, получаем:

$$ \chi'' - l * (l + 1) r^{-2} \chi = 0 $$

Ищем решение в виде $\chi = Cr^s$.
После подстановки получаем $s = -l, l+1$

Учитывая, что волновая функция имеет вид

$$\psi = \frac{\chi (r)}{r} Y_l^m (\theta, \phi)$$,

из требования непрерывности $\psi$ получим граничное условие

$$\psi (0) = 0$$

Следовательно $\chi (r) \approx Cr^{l+1}$ при $r \rightarrow 0$.