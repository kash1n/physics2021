\section{Соотношение неопределённостей Гейзенберга.}
\noindent{\bf Утверждение.} Несколько наблюдаемых $(F_1, F_2, ...)$ одновременно измеримы (имеют с достоверностью определённые значения для собственных состояний $|f\rangle$) тогда и только тогда, когда $[F_i,F_j] = 0,\ \forall i, j$. \note{(коммутатор равен нулю)}
\begin{proof}
Рассмотрим случай двух наблюдаемых.\\
Необходимость:
$$
\begin{aligned}
&F_1|f_1,f_2,...\rangle = f_1|f_1,f_2,...\rangle\\
&F_2|f_1,f_2,...\rangle = f_2|f_1,f_2,...\rangle\\
&......
\end{aligned}
\eqno(1)
$$
Необходимость очевидна из (1), если к первому соотношению оператор $F_2$, ко второму $F_1$ и произвести вычитание, то получим желаемое.\\
Достаточность:\\
Предположим, что в спектре оператора $F_1$ каждому собственному значению отвечает единственный собственный вектор (т.е. спектр простой).
$$
F_2 (F_1 |f_1,f_2\rangle) = f_1 F_2 |f_1,f_2\rangle = F_1 (F_2 |f_1,f_2\rangle)
$$
$=>\ F_2|f_1,f_2\rangle$ -- с.в. оператора $F_1$ с с.з. $f_1$.\\
В силу простоты спектра $F_1$, $F_2|f_1,f_2\rangle = \text{const} |f_1, f_2\rangle$.
Полагая $\text{const} = f_2$, получим, что вектор также собственный для $F_2$.\\
Поскольку система с.в. $F_1$ полна, то система общих с.в. $F_1$ и $F_2$ полна. В случае, если среди с.з. $F_1$ есть совпадающие, приведённое построение необходимо дополнить построением определённой линейной комбинации векторов, принадлежащих одному с.з.
\end{proof}
Рассмотрим пару некоммутирующих наблюдаемых: самосопряжённых операторов $A^+ = A,\ B^+ = B$. Тогда
$$
[A, B] = iC,
$$
где $C^+ = C$. Определим {\bf дисперсии} величин равенствами
$$
(\delta A)^2 = \langle(A-\langle A\rangle)^2\rangle,\ \ (\delta B)^2 = \langle(B-\langle B\rangle)^2\rangle,\ \ (\delta C)^2 = \langle(C-\langle C\rangle)^2\rangle,
$$
где усреднение проводится по выбранному состоянию $|\psi\rangle$. Тогда
$$
(\delta A)^2(\delta B)^2 \ge \frac{1}{4}\langle C\rangle^2
$$
Это ограничение называется {\bf соотношением неопределенностей} (СН) и получено впервые Гейзенбергом в 1927 г. для частного случая
наблюдаемых $x$ (координата) и $p_x$ (импульс):
$$
[x,p_x] = i\hbar I\ =>\ (\delta x)^2(\delta p_x)^2 \ge \frac{\hbar^2}{4}
$$
Это ограничение называют {\bf соотношением неопределённостей Гейзенберга}.\\
Для коммутирующих наблюдаемых правая часть СН обращается в нуль, что соответствует одновременной измеримости таких наблюдаемых.
Для некоммутирующих наблюдаемых СН накладывает ограничение на точности, с которыми могут быть одновременно заданы (измерены) эти наблюдаемые.\\
Найдем состояния $\psi$, в которых достигается минимум неопределенностей, т. е. точное равенство в СН. Получаем для них систему уравнений:
$$
\begin{aligned}
&(\lambda a - ib)\psi = 0,\\
&\lambda^2\langle a^2\rangle + \lambda \langle C\rangle + \langle b^2\rangle = 0,\\
&\langle a^2\rangle\langle b^2\rangle = \frac{1}{4}\langle C\rangle^2.
\end{aligned}
$$
Отсюда находим
$$
\lambda = -\frac{\langle C\rangle}{2\langle a^2\rangle}
$$
и уравнение для определения {\bf состояния, минимизирующего произведение неопределённостей}, принимает вид
$$
\left(\frac{\langle C\rangle}{2\langle a^2\rangle}a + ib\right)\psi = 0.
$$