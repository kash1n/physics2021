\section {Наблюдаемые и операторы.}
Наблюдаемая - физ. величина, значение которой может быть измерено.
Собственные значения физической величины, характеризующей состояние системы $f$: $f_0, f_1, ...$. Их совокупность - спектр собственных значений $f$.
Описание состояния системы осуществляется заданием функции $\psi(q)$.
Собственные волновые функции: $\psi_0, \psi_1, ...$. $\psi_i$ - собственные функции $f$ с соответствующими собственными значениями.\\
Условние нормировки: $\int |\psi_i|^2dq = 1$. Из принципа суперпозиции имеем для произвольного состояния $\psi$:
\begin{itemize}
\item для дискретного спектра:  $\psi = \Sigma_i c_i\psi_i$
\item для недискретного спектра: $\psi = \int c_f\psi_fdf$
\end{itemize}
Таким образом любая волновая функция может быть разложена по собственным функциям физической величины. \{$\psi_i$\} - полная система функций.\\
$|c_i|^2$ определяет вероятность $f_i$ в состоянии $\psi$. Следовательно, $\Sigma |c_i|^2 = 1$.\\
Среднее значение $f$: $\bar{f} = \Sigma |c_i|^2f_i$.\\
Рассмотрим оператор $\hat{f}$: $\bar{f} = \int \psi^*(\hat{f}\psi)dq$, следовательно он линейный. Таким образом мы сопоставили любому наблюдаемому в кв. мех. определенной линейный оператор.
\begin{center}
$\bar{f} = \int \psi_i^*(\hat{f}\psi_i)dq = f_i$, следовательно $\hat{f}\psi_i = f_i\psi_i$.
\end{center}
Собственные значения и средние значения вещественной физической величины в любом состоянии вещественны. Оператор $\hat{f}$ - эрмитов. Также имеем, что $\int \psi_i \psi_j^* = \delta_{ij}$. То есть с.ф. взаимноортогонольны, поэтому \{$\psi_i$\} - полная система ортонормированных функций.\\
