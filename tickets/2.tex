\section {Волновая функция.}
В экспериментах по дифракции пучка электронов проявляются волновые свойства электронов, причем аналогия с дифракцией электромагнитных волн, рассматриваемых как поток фотонов, приводит к статистическому предположению: интенсивность волны в данной точке пространства пропорциональна плотности частиц. Оказывается, что дифракционная картина не зависит от интенсивности пучка частиц: она возникает и при очень малой интенсивности и даже при пропускании одиночных электронов. Таким образом мы пришли к вероятностной интерпретации волновой функции частицы: величина $|\psi(t, {\bf r})|^2$ -- это плотность вероятности обнаружить частицу в точке пространства r в момент времени t. Обозначим $\psi(q, t) = \psi(t, {\bf r})$, где q - совокупность координат квантовой системы. За $dq$ обозначим элемент объема конфигурационного пр-ва.\\
Состояние системы может быть описано определенной комплекснозначной волновой функцией $\psi(q)$. Тогда величина $|\psi|^2dq$ показывает вероятность, что произведенное измерение над системой обнаружит частицу в элементе $dq$ конфигурационного пр-ва. \\
Запишем два УШ для $\psi, \psi^*$:
\begin{center}
$i\hbar\frac{\partial \psi     }{\partial t} = U\psi     - \frac{\hbar^2}{2m}\nabla^2\psi     = 0$\\
$i\hbar\frac{\partial \psi^*}{\partial t} = U\psi^* - \frac{\hbar^2}{2m}\nabla^2\psi^*= 0$, откуда получаем:\\
$\frac{\partial}{\partial t}(\psi\psi*) = -\frac{\hbar}{2mi}\nabla(\psi*\nabla\psi - (\nabla\psi*)\psi)$
\end{center}
Введем плотность и потом вероятности: $\rho, {\bf j}$: $\rho = |\psi|^2, {\bf j} = \frac{\hbar}{2mi}\nabla(\psi*\nabla\psi - (\nabla\psi*)\psi)$. Находим уравнение непрерывности: $\frac{\partial \rho}{\partial t} + \nabla \cdot {\bf j} = 0$. Проинтегрируем по объему V, ограниченному поверхностью $\Sigma$: $\frac{d}{dt}\int_V\rho d^3x = - \oint_{\Sigma}({\bf j \cdot n})d\Sigma$. В предположении $\Sigma \rightarrow \infty; {\bf j}\rightarrow 0$, получим $\int|\psi|^2d^3x = const $. Откуда можно всегда получить условие нормировки: $\int |\psi|^2dq = 1$. Вероятность найти частицу во всем пространстве равна единице, как и должно быть.\\
Заметим, что $\rho, {\bf j}$ инварианты относительно преобразования волновой функции на фазовый множитель $e^{i\alpha}$: $\psi \rightarrow \psi' = e^{i\alpha}\psi; \psi^* \rightarrow \psi'* = e^{i\alpha}\psi^*$. Функции $\psi, \psi'$ отвечают одному состоянию. Пусть $\psi = \sqrt{\rho}e^{i\theta}$. Тогда ${\bf j} = \frac{\hbar}{m}\rho\nabla\theta$. Для частицы с энергией E и импульсом {\bf p} имеем:
\begin{center}
$\psi = A\exp{[-\frac{i}{\hbar}(Et - {\bf p \cdot r})]}, {\bf j} = |A|^2\frac{{\bf p}}{m} = \rho{\bf v}$
\end{center}