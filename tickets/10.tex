\section {Уравнение Паули.}
\subsection {Тождественные частицы}
Спин в аппарат квантовой механики был введен Паули. Он
предложил \textbf{постулировал} для описания электрона уравнение,
которое теперь называется \textbf{уравнением Паули}:


$$ i\hbar\frac{\partial\psi}{\partial t}=\hat{H}_P\psi$$
$$ \hat{H}_P = \frac{1}{m_e} \left( \hat{p} - \frac{e}{c}A \right) + e\Phi - \hat \mu \hat B$$, 

Паулиевский гамильтониан $\hat{H}_P $ отличается от шрёдингеровского
добавлением слагаемого $\hat U_p = - \hat \mu \hat B$
, описывающего взаимодействие с магнитным полем
$B = \nabla \times A$.
спинового магнитного момента
электрона, представляемого оператором.
$\hat \mu$ - собственный магнитный момент частицы.

Рассмотрим «вывод» уравнения Паули, принадлежащий
Фейнману (R. P. Feynman). Из основного соотношения для матриц
Паули:

$$\sigma_k \sigma_n = \delta_{kn} + i \epsilon_{kns} \sigma_{s},$$

Следует

$$(\sigma \cdot a) (\sigma \cdot b) = a \cdot b + i \sigma \cdot (a \times b),$$

где $a$, $b$ - произвольные вектора. Учитывая это запишем гамильтониан электрона в электростатическом поле $U = e \Phi$ в эквивалетном шрёдингеровскому виде


$$ \hat{H} = \frac{(\sigma \cdot \hat p)^2} {2m_e} + e\Phi$$

Введем взаимодействие с магнитным полем (это постулат)

$$\hat p \rightarrow \hat P = \hat p  - \frac{e}{c} A$$

Тогда получим гамильтониан


$$ \hat{H}_F = \frac{(\sigma \cdot \hat P)^2} {2m_e} + e\Phi, $$

который эквивалентен гамильтониану Паули $\hat H_F = \hat H_P$. Действительно, 

$$ (\sigma \cdot \hat P)^2 = \hat P^2 + i \sigma \cdot (\hat P \times \hat P),$$

где второе слагаемое отлично от нуля ввиду некоммутативности
компонент оператора кинетического импульса $\hat p$:

$$\left[ \hat P_n, \hat P_k \right] = [\hat p_n - \frac{e}{c}A_n, \hat p_k - \frac{e}{c}A_k] = i \hbar \frac{e}{c}(\partial_n A_k - \partial_k A_n)$$

Следовательно,

$$(\hat P \times \hat P)_s = \epsilon_{snk}\hat P_n \hat P_k = \frac{1}{2} \epsilon_{snk} [\hat P_n, \hat P_k] = i \hbar \frac{e}{c} \epsilon_{snk} \partial_n A_k$$ 

В результате получаем 
$$\frac{(\sigma \cdot \hat P)^2}{2m_e} = \frac{\hat P^2}{2m_e} - \frac{e \hbar}{2m_e c} \sigma \cdot B,$$

т. е. приходим к паулиевскому взаимодействию спинового
магнитного момента электрона с магнитным полем.