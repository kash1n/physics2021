\section {Термодинамические параметры и потенциалы.}
Билет слишком большой и я бы рекомендовала убрать по вашему желанию из него все, что написано серым цветом.
\subsection{Температура}
Физические величины, характеризующие макроскопические
состояния тел, называют \textbf{термодинамическими}.\\
\textcolor{gray}{
\textbf{Энтропия} - макроскопическая величина, максимальное значение которой характеризует равновесие системы: $S = -<\ln \omega_n> = \sum_{n} \omega_n \ln \omega_n$. Это безразмерная величина.\\
}
\textbf{(Абсолютная)температура T}: $$\frac{1}{T} = \frac{dS}{dE}$$ E - энергия (в эргах)\\
S - энтропия (безразмерная)\\
\textbf{Постоянная Больцмана} - переводной коэффициент между эргами и
градусами, т. е. число эргов на градус \\$k = 1,380649 \cdot 10^{-23} $ Дж/град $ = 1,380649 \cdot 10^{-16}$ эрг/град\\
\textbf{\underline{Утверждение:} Температура тел, находящихся в равновесии друг с другом, одинаковы.}\\
\underline{Доказательство:}
Рассмотрим два тела, находящиеся в тепловом равновесии
друг с другом, причем оба тела вместе составляют замкнутую
систему. Тогда энтропия S этой системы имеет наибольшее воз­можное (при данной энергии Е системы) значение.
$$
E = E_1 + E_2
$$
$$
S = S_1(E_1) + S_2(E_2)
$$
Поскольку $E_2 = E + E_1$, где E - постоянная, то S есть в действительности функция одной независимой пе­ременной, и необходимое условие максимума можно написать в виде:
$$
\frac{dS}{dE_1} = \frac{dS_1}{dE_1} + \frac{dS_2}{dE_2}\frac{E_2}{dE_1} = \frac{dS_1}{dE_1} - \frac{dS_2}{dE_2} = 0
$$
Отсюда
$$
\frac{dS_1}{dE_1} = \frac{dS_2}{dE_2}
$$
Этот вывод без труда обобщается на случай любого числа тел,
находящихся в равновесии друг с другом.\\
Таким образом, если система находится в состоянии термо­динамического равновесия, то производная энтропии по энер­гии для всех ее частей одинакова, т. е. постоянна вдоль всей
системы.Температуры тел, находящихся в равновесии друг с другом, сле­довательно, одинаковы.\\
\subsection{Давление}
\textcolor{gray}{
Найдем силу, с которой тело действует на границу своего объема. Согласно известным формулам механики сила, действующая на некоторый элемент поверхности ds, равна:
$$
\mathbf{F} = - \frac{\partial E(p,q, \mathbf{r})}{\partial \mathbf{r}}
$$
где E - энергия тела как функция координат и импуль­сов его частиц, а также радиуса-вектора данного элемента по­верхности, играющего в данном случае роль внешнего пара­метра.\\
Усредним:
$$
\overline{\mathbf{F}} = - \frac{\overline{\partial E(p,q, \mathbf{r})}}{\partial \mathbf{r}} = - (\frac{\partial E}{\partial \mathbf r})_S =- (\frac{\partial E}{\partial V})_S \frac{\partial V}{\partial \mathbf r}
$$
тут воспользовались формулой, позволяющей вычислить термодинамическим путем средние (по равновесному статистическому распределению) (формула 11.3 из [Л2]).\\
Так как изменение объема равно $d\mathbf s d\mathbf r$, где $d\mathbf s$ - элемент поверхности, то $\frac{\partial V}{\partial \mathbf r} = d\mathbf s$ и поэтому
$$
\overline{\mathbf{F}} = - (\frac{\partial E}{\partial V})_S d\mathbf s
$$\\
Отсюда видно, что средняя сила, действующая на элемент по­верхности, направлена по нормали к этому элементу и пропор­циональна его площади (\textbf{закон Паскаля}).\\
}
\textbf{Давление} - абсолютная величина силы, действующей на единицу площади поверхности:
\begin{equation}\label{e0}
P = - (\frac{\partial E}{\partial V})_S
\end{equation}
\textcolor{gray}{
При определении температуры формулой выше речь шла по существу о теле, непосредственно не соприкасающемся ни с ка­кими другими телами и, в частности, не окруженном никакой внешней средой. В этих условиях можно было говорить об из­менении энергии и энтропии тела, не уточняя характера про­цесса. В общем же случае тела, находящегося во внешней среде (или окруженного стенками сосуда), формула должна быть
уточнена.\\
Действительно, если в ходе процесса меняется объем
данного тела, то это неизбежно отразится и на состоянии со­
прикасающихся с ним тел, и для определения температуры надо
было бы рассматривать одновременно все соприкасающиеся те­ла (например, тело вместе с сосудом, в который оно заключено).
Если же мы хотим определить температуру по термодинамиче­ским величинам одного только данного тела, то надо считать
объем этого тела неизменным. Другими словами, температура
определится как:\\
}
Формула для температуры при неизменном объеме:
\begin{equation}\label{e1}
T = (\frac{\partial E}{\partial S})_V
\end{equation}
из двух предыдущих формул получаем \textbf{первое начало термодинамики}:
$$
dE = TdS - PdV
$$
\textbf{\underline{Утверждение:} Давления тел, находящихся в равновесии друг с другом, одинаковы.}\\
\underline{Доказательство:} рассмотрим две сопри­касающиеся части находящейся в равновесии замкнутой систе­мы. Одним из необходимых условий максимальности энтропии является условие максимальности по отношению к изменению объемов $V_1$ и $V_2$ этих двух частей при неизменных состояниях остальных частей; последнее означает, в частности, что остается неизменной и сумма $V_1 + V_2$. Имеем:
$$
\frac{dS}{dV_1} = \frac{dS_1}{dV_1} + \frac{dS_2}{dV_2}\frac{V_2}{dV_1} = \frac{dS_1}{dV_1} - \frac{dS_2}{dV_2} = 0
$$
Перепишем первое начало термодинамики в виде:
$$
dS = \frac{1}{T}dE+ \frac{P}{T}dV
$$
Отсюда видно, что $\partial S / \partial V = P / T$. Так что $P_1/T_1 = P_2/T_2$ Так как температуры при равновесии одинаковы, то мы получаем
отсюда искомое равенство давлений $P_1 = P_2$.

\subsection{Работа и количество тепла}
\textbf{Работа}, произведенная над телом
при изменении его объема (отнесенная к единице времени):
$$
\frac{dR}{dt} = -P\frac{dV}{dt}
$$
\textcolor{gray}{
(в течение всего процесса тело должно находиться в состоянии механическо­го равновесия, т. е. в каждый момент времени давление должно
быть постоянным вдоль всего тела.)}\\
Если тело теплоизолировано, то все изменение его энергии
связано с производимой над ним работой. В общем же случае
нетеплоизолированного тела, помимо работы, тело получает (или
отдает) энергию и путем непосредственной передачи от других
соприкасающихся с ним тел. Эта часть изменения энергии на­зывается количеством полученного (или отданного) телом \textbf{теп­ла} Q. Таким образом, \textbf{изменение энергии тела (в единицу време­ни)} можно написать в виде:
$$
\frac{dE}{dt} = \frac{dR}{dt} + \frac{dQ}{dt} = -P\frac{dV}{dt} + \frac{dQ}{dt}
$$
Первое начало термодинамики можно переписать в виде:
$$
\frac{dE}{dt} = T\frac{dS}{dt}-P\frac{dV}{dt}
$$
Тогда
$$
\frac{dQ}{dt} = T\frac{dS}{dt}
$$
\textbf{Теплоемкость} - количество тепла, при получении которого температура те­ла повышается на единицу температуры.\\
\textcolor{gray}{
Обычно различают теплоемкость $C_v$ при постоянном объеме и теплоемкость $C_p$ при постоянном давлении:
$$
C_v = T (\frac{\partial S}{\partial t})_V
$$
$$
C_p = T (\frac{\partial S}{\partial t})_P
$$
}

\subsection{Свободная энергия и термодинамический потенциал}
Работу, произведенную над телом при бесконечно малом
изотермическом обратимом изменении его состояния, можно
написать в виде дифференциала некоторой величины
$$
dR-dQ-TdS=d(E-TS)
$$
или 
$$
dR=dF
$$
 $F = E-TS$ - \textbf{Свободная энергия}.\\
Таким образом, работа, производимая над телом при
обратимом изотермическом процессе, равна изменению его сво­бодной энергии. \\
Найдем дифференциал свободной энергии.
Подставим
$$
dE = TdS - PdV
$$
в
$$
dF=  dE - TdS - SdT
$$
Получим 
$$
dF = -SdT-PdV
$$
Отсюда следуют равенства:
\begin{equation}\label{e3}
S = - (\frac{dF}{dT})_V
\end{equation}
\begin{equation}\label{e4}
P = - (\frac{dF}{dV})_T
\end{equation}
Пользуясь соотношением Е = F -TS, можно выразить энергию
через свободную энергию в виде
$$
E = F - T (\frac{dF}{dT})_V = -T^2(\frac{\partial}{\partial T}\frac{F}{T})_V
$$
Формулы \ref{e0}, \ref{e1}, \ref{e3}, \ref{e4} \textcolor{gray}{и $ T = (\frac{dW}{dS})_P, V = (\frac{dW}{dP})_S$ (W- тепловая функция, которую ввели в 14.2 [Л2])} показывают, что, \textbf{зная какую-либо из величин Е\textcolor{gray}{, W} или F (как функцию соответствую­щих двух переменных) и составляя ее частные производные, можно определить все остальные термодинамические величины.}\\
По этой причине величины E\textcolor{gray}{, W}, F называют вообще \textbf{термо­динамическими потенциалами} (по аналогии с механическим по­тенциалом) или характеристическими функциями: энергию Е —
по отношению к переменным S, V , \textcolor{gray}{тепловую функцию W —
по отношению к S, Р ,} свободную энергию F — по отношению
к V, Т.\\
У нас не хватает еще термодинамического потенциала по от­
ношению к переменным Р , Т.\\
Для его получения подставляем в
$$
dF = -SdT-PdV
$$
$$
PdV = d(PV) - VdP
$$
Получим 
$$
d\Phi = -SdT+VdP
$$
где вводится \\
\textbf{Термодинамический потенциал} $\Phi = E - TS +PV=F+PV$\textcolor{gray}{=W-TS}\\
Отсюда имеем равенства
$$
S = -(\frac{\partial\Phi}{\partial T})_P
$$
$$
V = (\frac{\partial\Phi}{\partial P})_T
$$

