
\section{Распределение Планка. Фотонный газ.}

Применим статистику Бозе к электромагнитному излучению, находящемуся в термодинамическом равновесии: черному излучению (это газ, состоящий из фотонов).
Условие идеальности требует малости взаимодействия излучения с веществом (но в газах с бОльшей плотностью нужны высокие температуры). Но наличие вещества необходимо: равновесие достигается через испускание и поглощение фотонов веществом. Число частиц N -- переменно определяется через условие равновесия. Из минимальности совбодной энергии $\frac{\partial F}{\partial N} = 0$, но ${\frac{\partial F}{\partial N}}_{T,V} = \mu$, следовательно $\mu = 0 $, где $\mu$ -- химический потенциал.
\\
Распределение фотонов по различным квантовым состояниям с определенными значениями импульса $\overline{h}\overline{k}$ и энергии $\epsilon = \overline{h}\omega = \overline{h}cK$, $\mu = 0$:
$$\overline{n}_k = \frac{1}{e^{\frac{\overline{h}\omega}{T}}-1}$$ - распределние Планка.
\\
С достаточно большим объемом переходим к непрерывному распределению собственных частот излучения:
$$V\frac{d^3k}{(2\pi)^3}$$
\\
$N_{\omega}$ -- число фотонов в заданном интервале частот, второй знаментаель из распределения Планка:

$$dN_{\omega} = \frac{V}{\pi^2c^3} \frac{\omega^2d\omega}{e^{\frac{\overline{h}\omega}{T}}-1} | \overline{h \omega}$$
$$fE_{\omega} = \frac{V\overline{h}}{\pi^2c^3} \frac{\omega^2d\omega}{e^{\frac{\overline{h}\omega}{T}}-1}$$ -- формула Планка (формула спектрального распределения энергии черного излучения), где $E_{\omega}$ --энергия излучения, заключенная в участке спектра.
\\
При малых частотах $\overline{h}\omega \ll T $ формула Релея-Джинса:
$$dE_{\omega} = V\frac{T}{\pi^2c^3}\omega d\omega$$
При больших частотах $\overline{h}\omega \gg T $ формула Вина:
$$dE_{\omega} = V\frac{\overline{h}}{\pi^2c^3}\omega^3 e^{-\frac{\overline{h}\omega}{T}} d\omega$$
Закон смещения: при повышении т-ру положение максимума распределения смещается в сторону больших частот пропорционально Т (плотность спектрального распределения имеет максимум при $\omega_m: \frac{\overline{h}\omega_m}{T} = 2.922$).
\\
Термодинамические величины:
\\
Свободная энергия:
$$
F = N_{\mu} + \Omega = \Omega = T \frac{V}{\pi^2c^3} \int\limits_{0}^{\infty} \omega^2 ln(1- e^{\frac{-\overline{h}\omega}{T}}), d\omega = |x = \frac{\overline{h}\omega}{T}, делаем замену интегрируем по частям| =
$$ 
$$
= -V \frac{T^4}{3\pi^2\overline{h}^3 c^3} \int\limits_{0}^{\infty}\frac{x^3dx}{e^x-1} = - V\frac{\pi^2T^4}{4\pi(\overline{h}c)^3} = -\frac{4\sigma}{3c}VT^4
$$
Тут сигма - постоянная Стефана-Больцмана ($\frac{\pi^2k^4}{60\overline{h}^3c^2}$).
\\
Энтропия:
$$S = -\frac{\partial F}{\partial T} = \frac{16\sigma}{3c}VT^3$$
Полная энергия излучения (закон Больцмана: полная энергия черного излучения пропорциональна четвертой степени температуры):
$$E = F + TS = \frac{4\sigma}{c}VT^4 = -3F$$
Теплоемкость: 
$$C_{\nu} = (\frac{\partial E}{\partial T})_{\nu} = \frac{16\sigma}{c}T^3V$$
Давление:
$$p = - (\frac{\partial F}{\partial V})_T = \frac{4\sigma}{3c}T^4$$
Полное число фотонов:
$$N = \frac{V}{\pi^2c^3} \int\limits_{0}^{\infty} \frac{\omega^2d\omega}{e^{\frac{\overline{h}\omega}{T}}-1} = \frac{VT^3}{\pi^2c^3\overline{h}^3} \int\limits_{0}^{\infty} \frac {x^2dx}{e^x - 1} = 0.244(\frac{T^3}{\overline{h}c})V$$

