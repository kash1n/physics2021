\section{Распределение Бозе–Эйнштейна. Идеальный бозе-газ.}
\note{Я не уверена, что это то, что нужно, но мои глаза вытекают при перетехивании от руки написанных лекций, поэтому соре. Для желающих: 5 с конца страница лекций}

Очень интересным и важным примером применения статистики Бозе– Эйнштейна является теория так называемого излучения абсолютно черного тела (равновесного теплового излучения). Именно решение этой задачи М. Планком послужило началом квантовой физики. Представим себе полость (камеру), имеющую фиксированный объем V , температура которой T поддерживается постоянной.
Будем считать, что внутренняя поверхность каме- ры абсолютно черная, то есть она может изотроп- но поглощать и, соответственно, испускать элек- тромагнитное излучение любых частот от нуля до бесконечности
Будем рассматривать излучение как фотонный газ, находящийся в термодинамическом равновесии со стенками камеры. Так как фотоны очень слабо взаимодействуют друг с другом, то для их описания хорошо работает модель идеального газа.
\\
Число фотонов в полости не фиксировано, однако задана температура системы. Состояние равновесия и, как следствие, среднее число фотонов в полости, проще всего найти из условия максимума энтропии.
$$
S =k \sum\limits_{p} \{ (1+\langle n_p \rangle)ln(1+\langle n_p \rangle)-\langle n_p\rangle ln\langle n_p \rangle \}
$$

Будем искать ее условный максимум, считая, что числа заполнения нормированы только соотношением
$$E = \sum\limits_{p} \varepsilon_p n_p$$
Введем функцию 
$$
S/k = \sum\limits_{p} \{ (1+ n_p)ln(1+ n_p)- n_p ln n_p \} - \beta(\sum\limits_{p}\varepsilon_pn_p - E)
$$
где $\beta$ неопределенный множитель. Найдем вариацию функции и приравняем ее нулю:
$$ln\frac{1+n_p}{n_p} - \beta\varepsilon_p = 0$$
Следовательно,
$$n_p = \frac{1}{e^{\beta\varepsilon_p} - 1}$$
Таким образом, $\beta = \frac{1}{kT}$ и мы получили распределение Бозе–Эйнштейна с нулевым химическим потенциалом $\mu = 0$
Химический потенциал в этом случае строго равен нулю.
