\section{Принцип суперпозиции.}
%{\bf Из Гальцова:}\\
%Здесь и далее используются скобочные обозначения Дирака: вектор состояния (кет-вектор) обозначается скобкой $|\psi\rangle$, сопряженный вектор (бра-вектор) -- $\langle\psi|$, а скалярное призведение $(\psi, \varphi) = \langle\psi|\varphi\rangle$. В этих обозначениях скалярное произведение образуется всякий раз, когда бра-вектор стоит слева от кет-вектора. Норма вектора есть $||\psi|| = \left(\langle\psi|\psi\rangle\right)^{1/2}$\\

Уравнение Шрёдингера линейно, операторы наблюдаемых линейны $=>$ выполняется {\bf принцип суперпозиции:}
\begin{enumerate}
\item Если квантовая система может находиться в состояниях, описываемых волновыми функциями $\psi_1$ и $\psi_2$, то она может находиться и в состоянии $\psi = C_{1}\psi_1 + C_{2}\psi_2$, где $C_1, C_2 \in \mathbb{C}$.
\item $\psi$ и $C\psi,\ C\in\mathbb{C}$ описывают одно и то же состояние.
\end{enumerate}

Таким образом, для физически реализуемых состояний $||\psi|| < \infty$ можно выбрать $||\psi|| = 1$.

\subsection*{Общее решение уравнения Шрёдингера}
Рассмотрим стационарное уравнение Шрёдингера (потенциальная энергия не зависит от $t$)
$$
  \nabla^2\psi + \frac{2m_0}{\hbar^2}\left(E - V(\overline{r})\right)\psi = 0
$$
Когда мы предъяляем требования к $\psi$ (непрерывность, непрерывность производной, однозначность и т.п.), решение существует лишь при определённом значении параметра: $E = E_1, E_2, ...$ -- {\bf энергетические уровни} системы. Решения -- {\bf собственные функции} $\psi_1, \psi_2, ...$
Частные решения имеют вид 
$$
\psi_n (\overline{r}, t) = e^{-\frac{i}{\hbar}E_{n}t}\psi_n(\overline{r}),\ \ \psi^{*}_{n} (\overline{r}, t) = e^{-\frac{i}{\hbar}E_{n}t}\psi^{*}_{n}(\overline{r})
$$
Из линейности уравнения и принципа суперпозиции следует, что общее решение есть линейная комбинация частных:
$$
\psi = \sum_{n}C_n e^{-\frac{i}{\hbar}E_{n}t}\psi_n,\ \ \psi^{*} = \sum_{n}C^{*}_{n} e^{-\frac{i}{\hbar}E_{n}t}\psi^{*}_{n}
$$
Условие нормировки записывается как
$$
\sum_{n, n'}C_{n'}^{*}C_n e^{-\frac{i}{\hbar}t(E_n - E_{n'})}\int\psi_{n'}^{*}\psi_{n}d^{3}x = 1
$$
С условием ортонормированности $\int\psi^{*}_{n'}\psi_{n}d^3x = \delta_{nn'}$ получаем 
$$
\sum_{n}C^{*}_{n}C_n = 1
$$
Тогда интерпретация такова: $|C_n|^2$ характеризует вероятность нахождения частицы в состоянии $n$ (напомним, $|\psi_n|^2$ трактуется как плотность вероятности распределения по пространству частицы, находящейся в состоянии $\psi_n$).

\subsection*{Ансамбли}
\noindent{\bf Квантовые/чистые:} сумма для волновых функций.\\
Совокупность невзаим. друг с другом частиц, которые могут находиться в состоянии $n_1$ или $n_2$:
$$
\psi = C_{n_1}e^{-\frac{i}{\hbar}E_{n_1} t}\psi_{n_1} + C_{n_2}e^{-\frac{i}{\hbar}E_{n_2} t}\psi_{n_2}
$$ 
$$
\psi^{*}\psi = C_{n_1}^{*}C_{n_1}\psi^{*}_{n_1}\psi_{n_1} + C_{n_2}^{*}C_{n_2}\psi^{*}_{n_2}\psi_{n_2} + C_{n_2}^{*}C_{n_1}e^{-\frac{i}{\hbar}t(E_{n_1} - E_{n_2})}\psi_{n_2}^{*}\psi_{n_1} + C_{n_1}^{*}C_{n_2}e^{-\frac{i}{\hbar}t(E_{n_1} - E_{n_2})}\psi_{n_1}^{*}\psi_{n_2}
$$
Смешанные члены с $C_{n_2}^{*}C_{n_1}$ и $C_{n_1}^{*}C_{n_2}$ определяют {\bf статистическую связь} между невзаим. частицами в разных состояниях.\\
Плотность вероятности $|\psi|^2 = |C_1|^2|\psi_1|^2 + |C_2|^2|\psi_2|^2 + 2\text{Re} C_1^{*}C_2\psi^{*}_1\psi_2$\\
Квадрат нормы $(\psi,\psi) = 1 = |C_1|^2 + |C_2|^2 + 2\text{Re} C_1^{*}C_2(\psi_1,\psi_2)$\\
Среднее значение наблюдаемой $\langle A\rangle = (\psi, \hat{A}\psi) = |C_1|^2(\psi_1,\hat{A}\psi_1) + |C_2|^2(\psi_2,\hat{A}\psi_2) + 2\text{Re} C_1^{*}C_2(\psi_2,\hat{A}\psi_2)$\\
Отсюда видно, что квантовая механика не сводится к классической теории вероятности: возникает характерный эффект интерференции состояний $\psi_1$ и $\psi_2$, не имеющий классического аналога.\\

\noindent{\bf Смешанные:} суммая для вероятностей $|C|^2 = |C_1|^2 + |C_2|^2$.\\
Нет статистической связи между различными состояниями и типичных волновых процессов (классическая теория).\\
В волновых процессах члены $C_2^{*}C_1$ и $C_1^{*}C_2$ могут исчезать когда фаза между различными квантовыми состояниями быстро изменяется (некогерентный свет).

\subsection*{Разложение состояний}
Состояния мы задаём как функционалы $\langle\omega|A\rangle$ со свойствами:
\begin{enumerate}
	\item $\langle\lambda A + B|\omega\rangle = \lambda\langle A|\omega\rangle + \langle B|\omega\rangle$
	\item $\langle A^2|\omega\rangle \ge 0$ (среднее значение $A^2$ неотрицательно)
	\item $\langle C|\omega\rangle = C$ (среднее значение $C$ совпадает с константой для любого состояния)
	\item $\langle A|\omega\rangle = \langle A|\omega\rangle$ (средние значения вещественны)
\end{enumerate}
Такие функционалы можно задать следующим образом:
$$
	\langle A|\omega\rangle = \text{Tr}MA,
$$
$M$ -- оператор в $\mathbb{C}^n$ такой, что $M^* = M,\ (M\xi,\xi)\ge 0,\ \text{Tr}M = 1$.\\
$M$ -- {\bf матрица плотности}.\\
Выпуклая комбинация $M = \alpha M_1 + (1-\alpha)M_2,\ 0<\alpha<1$, обладает теми же свойствами, что и соответствующее состояние $\omega = \alpha\omega_1 + (1-\alpha)\omega_2$.\\
Состояние, не раскладывающееся в выпуклую комбинацию других -- {\bf чистое}. Любое состояние является выпуклой комбинацией чистых состояний.