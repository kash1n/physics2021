\section {Уравнение Шрёдингера.}
\subsection*{Равновесие ЭМ излучения и вещества}
Замк сосуд, темп $T$, внутри него равновесное ЭМ излучение. Излучаемая и поглощаемая атомами вещества стенок сосуда в единицу времени энергии равны. {\bf Спектральная плотность энергии равновесного излучения} -- универсальная функция частоты и температуры $\rho(\omega, T) $. Из классической теории закон Рэлея:  $\rho(\omega, T) = \omega^2T$. Но тогда $\int_0^\infty \rho d\omega = \infty$

\subsection*{Корпускулярно-волновой дуализм (Гипотеза планка)}
{\bf Фундаментальная гипотеза квантования}: вещество испускает энергию излучения конечными порциями (квантами), пропорциональными частоте излучения. Коэффициент пропорциональности – универсальная постоянная $h$. Для простейшей модели вещества, в которой оно представляется в виде атомных осцилляторов, энергия осциллятора с частотой $\omega$ равна: $E_n = n\hbar \omega, n = 0,1,2...$. Здесь $\hbar = \frac{h}{2\pi}$ - это постоянная Планка.\\

 Из гипотезы квантования можно вывести, что
\begin{center}
$\rho(\omega, T) = \dfrac{\hbar \omega^3}{\pi^2c^3(e^{\hbar \omega/k_BT - 1})}$,
\end{center}
где $k_B$ - константа Больцмана, $c$ - скорость света.

\subsection*{Фотоэффект}
По Эйнштейну ЭМ излучение состоит из фотонов. Энергия фотона с частотой $\omega$ равна $\varepsilon = \hbar \omega$. {\bf Фотоэффект}: вещество излучает электроны под действием падающих на него высокочастотных фотонов. Для работы выхода для вещества есть соотношение: $\hbar \omega = \frac{mv^2}{2} + A$

\subsection*{Волновные св-ва эллектронов}
Л. де Бройль обобщил корпускулярно-волновой дуализм до электрона. Согласно ему частице с энергией и импульсом $E, p$ отвечает {\bf монохроматическая волна}, частота и волновой вектор которой завязаны соотношениями $\omega = \frac{E}{\hbar}, {\bf k} = \frac{{\bf p}}{\hbar}$. {\bf Дебройлевская длина волны частицы}: $\lambda = \frac{2\pi \hbar}{ p}$

\subsection*{Волновое уравнение Шредингера.}
Пусть волновое поле описывается { \it скалярной} функцией от времени $t$ и координат ${\bf r} = (x,y,z)$ -- волновой функцией $\psi (t, {\bf r})$. В случае монохроматической волны  $\psi (t, {\bf r}) = A\exp{[-i(\omega t - {\bf k \cdot r})]}$. В нашем курсе только нерелятивистская теория, т.е. энергия частицы массы $m$ равна $E = \frac{{\bf p}^2}{2m}$. Отсюда получаем зависимость частоты дебройлевской волны $\omega$ от волнового вектора ${\bf k}$ ({\bf Закон дисперсии}): $\omega = \frac{\hbar {\bf k}^2}{2m}$.\\
для монохроматической волны имеем: $\frac{\partial \psi}{\partial t} = -i\omega \psi, \nabla \psi = i{\bf k}\psi, \nabla^2\psi = -{\bf k}^2\psi $. Учитывая закон дисперсии, приходим к {\bf Уравнению Шредингера}(УШ) для свободной частицы:\\
\begin{center}
$i\hbar\frac{\partial \psi}{\partial t} = - \frac{\hbar^2}{2m}\nabla^2 \psi$.
\end{center}
Общее решение: $\psi (t, {\bf r})  = \int \dfrac{d^3k}{(2\pi)^{3/2}}C({\bf k})\exp{[-i(\omega t - {\bf k \cdot r})]}$.
\subsubsection*{Аналогия в классической механике}
Для свободной частицы имеем: $H = \frac{{\bf p}}{2m}, S(t,{\bf r}) = \int_0^tLdt, L = {\bf v \cdot p} - H, \nabla S = {\bf p}, \frac{\partial S}{\partial t} = - H$, что дает нам
{\bf уравнение Гамильтона-Якоби} (УГЯ). УГЯ выглядит так:
\begin{center}
$\frac{\partial S}{\partial t} + \frac{1}{2m}(\nabla S)^2 = 0$.
\end{center}
У уравнения есть интеграл $S = -Et+ S(t, {\bf r})$\\
Стационарное УГЯ: $E =  \frac{1}{2m}(\nabla S)^2 $\\
Устанавливаем связь с УШ: $S = -Et + {\bf p \cdot r} = -\hbar (\omega t - {\bf k \cdot r})$\\
Волновая функция выражается через S: $\psi = \exp{[\frac{iS}{\hbar}]}$\\
Подставим в УШ: $\frac{\partial S}{\partial t} + \frac{1}{2m}(\nabla S)^2 - \frac{i\hbar}{2m}\nabla^2S = 0$. Оно отличается от УГЯ только последним слагаемым, которое пропорционально $\hbar$. УШ для $\psi$ переходит в УГЯ для S только при $\hbar \rightarrow 0$


\subsection*{УШ в потенциальном поле U(t,{\bf r}).}
УГЯ: $\frac{\partial S}{\partial t} + \frac{1}{2m}(\nabla S)^2 + U = 0$\\
Квантовое обобщение: $\frac{\partial S}{\partial t} + \frac{1}{2m}(\nabla S)^2 + U - \frac{i\hbar}{2m}\nabla^2S = 0$ - нелинейное уравнение на $\psi$, но для объяснения диффракции и интерференции должно выполняться, что возмущение, создаваемое при прохождении ряда волн, должно быть суммой возмущений отдельных волн. Поэтому нужно линейное уравнение:
\begin{center}
$i\hbar\frac{\partial \psi}{\partial t} = \hat{H}\psi, \hat{H} =  - \frac{\hbar^2}{2m}\nabla^2 + U(t,{\bf r}) - $ оператор Гамильтона.
\end{center}
Получили {\bf уравнение Шредингера для частицы в потенциальном поле}.












































