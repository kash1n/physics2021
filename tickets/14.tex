\section {Каноническое распределение (распределение Гиббса).}
Рассмотрим макроскопическое тело, являющееся малой частью большой замкнутой системы. Будем рассматривать две части: тело и "всё остальное" (среду).\\
\textbf{Микроканоническое распределение} (вероятность нахождения системы в любом из $d\Gamma$ состояний):
$$
d\omega = const \cdot \delta (E + E' - E^{(0)})d\Gamma	d\Gamma'
$$
\begin{itemize}
	\item $d\Gamma$ - число квантовых состояний замкнутой системы, приходящихся на определенный бесконечно малый интервал значений её энергии
	\item $E, d\Gamma$ и $ E', d\Gamma'$ относятся соответственно к телу и среде
	\item $E^{(0)}$ - заданное значение энергии замкнутой системы; этому
	значению должна быть равна сумма $E + E'$ энергий тела и среды
\end{itemize}
Нашей целью является нахождение вероятности $\omega_n$ такого
состояния всей системы, при котором данное тело находится в
некотором определенном квантовом состоянии (с энергией $E_n$),
т. е. в состоянии, описанном микроскопическим образом. Ми­кроскопическим же состоянием среды мы при этом не инте­ресуемся, т. е. будем считать, что она находится в некотором макроскопически описанном состоянии.\\
Пусть $\Delta\Gamma'$ есть стати­стический вес макроскопического состояния среды; обозначим также через $\Delta E'$ интервал значений энергии среды, соответствующий интервалу $\Delta\Gamma'$ квантовых состояний.\\
Искомую вероятность $\omega_n$ мы найдем, заменив в предыдущем уравнении $d\Gamma$ еди­ницей, положив $E = E_n$ и проинтегрировав по $d\Gamma'$:
$$
\omega_n = const \cdot \int \delta(E_n + E' - E^{(0)})d\Gamma'
$$
Пусть $\Gamma'(E')$ - полное число квантовых состояний среды с энергией, меньшей или равной $E'$.\\
Поскольку подынтегральное выражение зависит только от $E'$, можно перейти к интегри­рованию по $dE'$, написав:
$$
d\Gamma' = \frac{d\Gamma'(E')}{dE'}dE'
$$
$$
\frac{d\Gamma'}{dE'} = \frac{e^{S'(E')}}{\Delta E'}
$$
где $S'(E')$ - энтропия среды как функция её энергии.\\
Тогда
$$
\omega_n = const \cdot \int \frac{e^{S'}}{\Delta E'}\delta(E_n + E' - E^{(0)})dE'
$$
Благодаря наличию $\delta$-функции интегрирование сводится к замене $E'$ на $E^{(0)} - E_n$.
Получаем
$$
\omega_n = const \cdot (\frac{e^{S'}}{\Delta E'})_{E' = E^{(0)} - E_n}
$$
Учтем теперь, что вследствие малости тела его энергия $E_n$ мала по сравнению с $E^{(0)}$. Величина $\Delta Е'$ относительно очень
мало меняется при незначительном изменении $E'$; поэтому в ней
можно просто положить $E' = E^{(0)}$, после чего она превратит­
ся в не зависящую от $E_n$ постоянную.\\
В экспоненциальном же множителе $e^{S'}$ надо разложить $S'(E^{(0)} - E_n)$ по степеням $E_n$ сохранив также и линейный член:
$$
S'(E^{(0)} - E_n) = S'(E^{(0)}) - E_n \frac{dS'	(E^{(0)})}{dE^{(0)}}
$$
Но производная от энтропии $S'$ о энергии есть не что иное,
как $1/T$, где $T$ - температура системы (температура тела и сре­ды одинакова, так как система предполагается находящейся в
равновесии).\\
Таким образом, получаем окончательно для $w_n$ следующее
выражение (\textbf{распределение Гибса} или \textbf{каноническое распределение}): 
$$
\fbox{$\omega_n = A e^{-\frac{E_n}{T}}$}
$$
где А - не зависящая от $E_n$ нормировочная постоянная.\\
Это — одна из важнейших формул статистики; она \textbf{определяет ста­тистическое распределение любого макроскопического тела,являющегося сравнительно малой частью некоторой большой замкнутой системы}.\\
Нормировочная постоянная А определяется условием $\sum \omega_n = 1$, откуда
$$
\frac{1}{A} = \sum_{n} e ^{-\frac{E_n}{T}}
$$
Среднее значение любой физической величины f , характеризую­щей данное тело, может быть вычислено с помощью распреде­ления Гиббса по формуле:
$$
f = \sum_{n} \omega_n f_{nn} = \frac{\sum_{n}f_{nn} e^{-\frac{E_n}{T}}}{\sum_{n} e^{-\frac{E_n}{T}}}
$$


