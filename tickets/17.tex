\section{Распределение Гиббса с переменным числом частиц.}

При выводе распределения Гиббса в 14 билете предполагалось, что число частиц в незамкнутой системе фиксировано. А на самом система может обмениваться частицами с термостатом (флуктуировать).\\
Вероятность того, что система имеет N частиц и находится в состоянии с полной энергией $E_{nN}$ пропорциональна числу конфигураций термостата с энергией $E_{term}= E_{tot} -E_{nN}$ и числом частиц  $N_{term}= N_{tot} -N$. Число конфигураций равно
$$
\Delta \Gamma_{tot} = exp [S_{term}(E_{tot} - E_{nN}, N_{tot}-N)] 
$$
(это все написано исходя из принципа равных вероятностей макроскопических состояний системы и термостата вместе).\\
Предполагая что $E_{nN} << E_{tot}, N << N_{tot}$:
$$
\Delta \Gamma_{tot} = const \cdot exp(-E_{nN}\frac{\partial S_{term}}{\partial E}-N \frac{\partial S_{term}}{\partial N})
$$
Из соотношения 
$$
dS = \frac{1}{T}(dE + PdV - \mu dN)
$$
(это вроде как первое начало термодинамики, но я не понимаю откуда член с dN)
следует, что 
$$
\frac{\partial S}{\partial E} = \frac{1}{T}
$$
$$
\frac{\partial S}{\partial N} = -\frac{\mu}{T}
$$
где $\mu$ - химический потенциал, Т - температура. Они одинаковы для системы и термостата.\\
В результате получаем \textbf{вероятность микросостояния системы с энергией $E_{nN}$ и числом частиц N} называемое \textbf{большое каноническое распределение} в виде
$$
\omega_{nN} = C \cdot exp(\frac{\mu N - E_{nN}}{T})
$$
где C - нормировочная постоянная, определяемая условием 
$$
	\sum_{N}\sum_{n}\omega_{nN} = 1 = C \sum_{N}exp (\frac{\mu N}{T}) \sum_{n}exp(-\frac{E_{nN}}{T})
$$
Суммирование выполняется сначала по всем квантовым состояниям при заданном N, далее по всем N.\\
Теперь можем вычислить энтропию:
$$
S = - \sum_{N}\sum_{n}\omega_{nN}\ln\omega_{nN} = - \ln C - \frac{\mu \overline{N}}{T} + \frac{\overline{E}}{T}
$$
где средние значения числа частиц и энергии:
$$
\overline{N} = 	\sum_{N}\sum_{n}N \omega_{nN}
$$
$$
\overline{E} = \sum_{N}\sum_{n}E_{nN} \omega_{nN}
$$
Эти величины следует отождествить с термодинамическими величинами: числом частиц и внутренней энергией тела. Опускаем символы средних значений и получаем:
$$
T\ln C = E - \mu N - TS = F - \mu N = \Omega
$$
где $\Omega$ - термодинамический потенциал, $\mu$ - химический потенциал.\\
Так что можем записать большое каноническое распределение в виде:
$$
\omega_{nN} = exp (\frac{\Omega + \mu N - E_{nN}}{T})
$$





