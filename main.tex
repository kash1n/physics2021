%physics2021.tex
\documentclass[specialist, subf, href, colorlinks=true, 12pt, times, mtpro, final]{disser}
\usepackage[utf8]{inputenc}
\usepackage[english,russian]{babel}
\usepackage{a4wide}
\usepackage{mathtext}
\usepackage{amsbsy}
\usepackage{amsthm}
\usepackage{amsmath}
\usepackage{amssymb}
\usepackage{amsfonts}
\usepackage{tikz}
\usepackage{verbatim}
\usepackage{graphicx}
\usepackage{xcolor}
\usepackage{hyperref}
 
 % Цвета для гиперссылок
\definecolor{linkcolor}{HTML}{543AE8} % цвет ссылок
\definecolor{urlcolor}{HTML}{543AE8} % цвет гиперссылок
\hypersetup{pdfstartview=FitH,  linkcolor=linkcolor,urlcolor=urlcolor, colorlinks=true}
 
 % Цвет для комментариев вида \note{...}
\definecolor{faded}{gray}{0.6}
\def\note{\textcolor{faded}}
 
\begin{document}

\tableofcontents

\newpage
\noindent {\Large \bf Программа экзамена по физике (весна 2021)}
\noindent Лектор: {\bf Борисов Анатолий Викторович}\\

\begin{enumerate}
{\footnotesize
\item Уравнение Шрёдингера.
\item Волновая функция.
\item Наблюдаемые и операторы.
\item Принцип суперпозиции.
\item Соотношение неопределённостей Гейзенберга.
\item Изменение наблюдаемых со временем.
\item Гармонический осциллятор.
\item Оператор момента импульса.
\item Спин.
\item Уравнение Паули.
\item Движение в центрально-симметричном поле.
\item Атом водорода.
\item Тождественные частицы. Принцип Паули.
\item Каноническое распределение (распределение Гиббса).
\item Термодинамические параметры и потенциалы.
\item Идеальный классический газ. Распределение Больцмана. 
\item Распределение Гиббса с переменным числом частиц. 
\item Распределение Ферми-Дирака. Идеальный ферми-газ.
\item Распределение Бозе-Эйнштейна. Идеальный бозе-газ.
\item Распределение Планка. Фотонный газ.
}
\end{enumerate}

%\section {Уравнение Шрёдингера.}
\note{текст билета}

 
\end{document}
