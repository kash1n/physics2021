%physics2021.tex
\documentclass[specialist, subf, href, colorlinks=true, 12pt, times, mtpro, final]{disser}
\usepackage[utf8]{inputenc}
\usepackage[english,russian]{babel}
\usepackage{a4wide}
\usepackage{mathtext}
\usepackage{amsbsy}
\usepackage{amsthm}
\usepackage{amsmath}
\usepackage{amssymb}
\usepackage{amsfonts}
\usepackage{tikz}
\usepackage{verbatim}
\usepackage{graphicx}
\usepackage{xcolor}
\usepackage{hyperref}
 
 % Цвета для гиперссылок
\definecolor{linkcolor}{HTML}{543AE8} % цвет ссылок
\definecolor{urlcolor}{HTML}{543AE8} % цвет гиперссылок
\hypersetup{pdfstartview=FitH,  linkcolor=linkcolor,urlcolor=urlcolor, colorlinks=true}
 
 % Цвет для комментариев вида \note{...}
\definecolor{faded}{gray}{0.6}
\def\note{\textcolor{faded}}
 
\begin{document}

\tableofcontents

\newpage
\noindent {\Large \bf Программа экзамена по физике (весна 2021)}
\noindent Лектор: {\bf Борисов Анатолий Викторович}\\

\begin{enumerate}
{\footnotesize
\item Уравнение Шрёдингера.
\item Волновая функция.
\item Наблюдаемые и операторы.
\item Принцип суперпозиции.
\item Соотношение неопределённостей Гейзенберга.
\item Изменение наблюдаемых со временем.
\item Гармонический осциллятор.
\item Оператор момента импульса.
\item Спин.
\item Уравнение Паули.
\item Движение в центрально-симметричном поле.
\item Атом водорода.
\item Тождественные частицы. Принцип Паули.
\item Каноническое распределение (распределение Гиббса).
\item Термодинамические параметры и потенциалы.
\item Идеальный классический газ. Распределение Больцмана. 
\item Распределение Гиббса с переменным числом частиц. 
\item Распределение Ферми-Дирака. Идеальный ферми-газ.
\item Распределение Бозе-Эйнштейна. Идеальный бозе-газ.
\item Распределение Планка. Фотонный газ.
}
\end{enumerate}

\section {Уравнение Шрёдингера.}
\note{текст билета}

\section {Волновая функция.}
В экспериментах по дифракции пучка электронов проявляются волновые свойства электронов, причем аналогия с дифракцией электромагнитных волн, рассматриваемых как поток фотонов, приводит к статистическому предположению: интенсивность волны в данной точке пространства пропорциональна плотности частиц. Оказывается, что дифракционная картина не зависит от интенсивности пучка частиц: она возникает и при очень малой интенсивности и даже при пропускании одиночных электронов. Таким образом мы пришли к вероятностной интерпретации волновой функции частицы: величина $|\psi(t, {\bf r})|^2$ -- это плотность вероятности обнаружить частицу в точке пространства r в момент времени t. Обозначим $\psi(q, t) = \psi(t, {\bf r})$, где q - совокупность координат квантовой системы. За $dq$ обозначим элемент объема конфигурационного пр-ва.\\
Состояние системы может быть описано определенной комплекснозначной волновой функцией $\psi(q)$. Тогда величина $|\psi|^2dq$ показывает вероятность, что произведенное измерение над системой обнаружит частицу в элементе $dq$ конфигурационного пр-ва. \\
Запишем два УШ для $\psi, \psi^*$:
\begin{center}
$i\hbar\frac{\partial \psi     }{\partial t} = U\psi     - \frac{\hbar^2}{2m}\nabla^2\psi     = 0$\\
$i\hbar\frac{\partial \psi^*}{\partial t} = U\psi^* - \frac{\hbar^2}{2m}\nabla^2\psi^*= 0$, откуда получаем:\\
$\frac{\partial}{\partial t}(\psi\psi*) = -\frac{\hbar}{2mi}\nabla(\psi*\nabla\psi - (\nabla\psi*)\psi)$
\end{center}
Введем плотность и потом вероятности: $\rho, {\bf j}$: $\rho = |\psi|^2, {\bf j} = \frac{\hbar}{2mi}\nabla(\psi*\nabla\psi - (\nabla\psi*)\psi)$. Находим уравнение непрерывности: $\frac{\partial \rho}{\partial t} + \nabla \cdot {\bf j} = 0$. Проинтегрируем по объему V, ограниченному поверхностью $\Sigma$: $\frac{d}{dt}\int_V\rho d^3x = - \oint_{\Sigma}({\bf j \cdot n})d\Sigma$. В предположении $\Sigma \rightarrow \infty; {\bf j}\rightarrow 0$, получим $\int|\psi|^2d^3x = const $. Откуда можно всегда получить условие нормировки: $\int |\psi|^2dq = 1$. Вероятность найти частицу во всем пространстве равна единице, как и должно быть.\\
Заметим, что $\rho, {\bf j}$ инварианты относительно преобразования волновой функции на фазовый множитель $e^{i\alpha}$: $\psi \rightarrow \psi' = e^{i\alpha}\psi; \psi^* \rightarrow \psi'* = e^{i\alpha}\psi^*$. Функции $\psi, \psi'$ отвечают одному состоянию. Пусть $\psi = \sqrt{\rho}e^{i\theta}$. Тогда ${\bf j} = \frac{\hbar}{m}\rho\nabla\theta$. Для частицы с энергией E и импульсом {\bf p} имеем:
\begin{center}
$\psi = A\exp{[-\frac{i}{\hbar}(Et - {\bf p \cdot r})]}, {\bf j} = |A|^2\frac{{\bf p}}{m} = \rho{\bf v}$
\end{center}
\section {Наблюдаемые и операторы.}
Наблюдаемая - физ. величина, значение которой может быть измерено.
Собственные значения физической величины, характеризующей состояние системы $f$: $f_0, f_1, ...$. Их совокупность - спектр собственных значений $f$.
Описание состояния системы осуществляется заданием функции $\psi(q)$.
Собственные волновые функции: $\psi_0, \psi_1, ...$. $\psi_i$ - собственные функции $f$ с соответствующими собственными значениями.\\
Условние нормировки: $\int |\psi_i|^2dq = 1$. Из принципа суперпозиции имеем для произвольного состояния $\psi$:
\begin{itemize}
\item для дискретного спектра:  $\psi = \Sigma_i c_i\psi_i$
\item для недискретного спектра: $\psi = \int c_f\psi_fdf$
\end{itemize}
Таким образом любая волновая функция может быть разложена по собственным функциям физической величины. \{$\psi_i$\} - полная система функций.\\
$|c_i|^2$ определяет вероятность $f_i$ в состоянии $\psi$. Следовательно, $\Sigma |c_i|^2 = 1$.\\
Среднее значение $f$: $\bar{f} = \Sigma |c_i|^2f_i$.\\
Рассмотрим оператор $\hat{f}$: $\bar{f} = \int \psi^*(\hat{f}\psi)dq$, следовательно он линейный. Таким образом мы сопоставили любому наблюдаемому в кв. мех. определенной линейный оператор.
\begin{center}
$\bar{f} = \int \psi_i^*(\hat{f}\psi_i)dq = f_i$, следовательно $\hat{f}\psi_i = f_i\psi_i$.
\end{center}
Собственные значения и средние значения вещественной физической величины в любом состоянии вещественны. Оператор $\hat{f}$ - эрмитов. Также имеем, что $\int \psi_i \psi_j^* = \delta_{ij}$. То есть с.ф. взаимноортогонольны, поэтому \{$\psi_i$\} - полная система ортонормированных функций.\\

 
\end{document}
