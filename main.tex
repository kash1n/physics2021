%physics2021.tex
\documentclass[specialist, subf, href, colorlinks=true, 12pt, times, mtpro, final]{disser}
\usepackage[utf8]{inputenc}
\usepackage[english,russian]{babel}
\usepackage{a4wide}
\usepackage{mathtext}
\usepackage{amsbsy}
\usepackage{amsthm}
\usepackage{amsmath}
\usepackage{amssymb}
\usepackage{amsfonts}
\usepackage{tikz}
\usepackage{verbatim}
\usepackage{graphicx}
\usepackage{xcolor}
\usepackage{hyperref}
 
 % Цвета для гиперссылок
\definecolor{linkcolor}{HTML}{543AE8} % цвет ссылок
\definecolor{urlcolor}{HTML}{543AE8} % цвет гиперссылок
\hypersetup{pdfstartview=FitH,  linkcolor=linkcolor,urlcolor=urlcolor, colorlinks=true}
 
 % Цвет для комментариев вида \note{...}
\definecolor{faded}{gray}{0.6}
\def\note{\textcolor{faded}}
 
\begin{document}

\tableofcontents

\newpage
\noindent {\Large \bf Программа экзамена по физике (весна 2021)}
\noindent Лектор: {\bf Борисов Анатолий Викторович}\\

\begin{enumerate}
{\footnotesize
\item Уравнение Шрёдингера.
\item Волновая функция.
\item Наблюдаемые и операторы.
\item Принцип суперпозиции.
\item Соотношение неопределённостей Гейзенберга.
\item Изменение наблюдаемых со временем.
\item Гармонический осциллятор.
\item Оператор момента импульса.
\item Спин.
\item Уравнение Паули.
\item Движение в центрально-симметричном поле.
\item Атом водорода.
\item Тождественные частицы. Принцип Паули.
\item Каноническое распределение (распределение Гиббса).
\item Термодинамические параметры и потенциалы.
\item Идеальный классический газ. Распределение Больцмана. 
\item Распределение Гиббса с переменным числом частиц. 
\item Распределение Ферми-Дирака. Идеальный ферми-газ.
\item Распределение Бозе-Эйнштейна. Идеальный бозе-газ.
\item Распределение Планка. Фотонный газ.
}
\end{enumerate}

\section {Уравнение Шрёдингера.}
\subsection*{Равновесие ЭМ излучения и вещества}
Замк сосуд, темп $T$, внутри него равновесное ЭМ излучение. Излучаемая и поглощаемая атомами вещества стенок сосуда в единицу времени энергии равны. {\bf Спектральная плотность энергии равновесного излучения} -- универсальная функция частоты и температуры $\rho(\omega, T) $. Из классической теории закон Рэлея:  $\rho(\omega, T) = \omega^2T$. Но тогда $\int_0^\infty \rho d\omega = \infty$

\subsection*{Корпускулярно-волновой дуализм (Гипотеза планка)}
{\bf Фундаментальная гипотеза квантования}: вещество испускает энергию излучения конечными порциями (квантами), пропорциональными частоте излучения. Коэффициент пропорциональности – универсальная постоянная $h$. Для простейшей модели вещества, в которой оно представляется в виде атомных осцилляторов, энергия осциллятора с частотой $\omega$ равна: $E_n = n\hbar \omega, n = 0,1,2...$. Здесь $\hbar = \frac{h}{2\pi}$ - это постоянная Планка.\\

 Из гипотезы квантования можно вывести, что
\begin{center}
$\rho(\omega, T) = \dfrac{\hbar \omega^3}{\pi^2c^3(e^{\hbar \omega/k_BT - 1})}$,
\end{center}
где $k_B$ - константа Больцмана, $c$ - скорость света.

\subsection*{Фотоэффект}
По Эйнштейну ЭМ излучение состоит из фотонов. Энергия фотона с частотой $\omega$ равна $\varepsilon = \hbar \omega$. {\bf Фотоэффект}: вещество излучает электроны под действием падающих на него высокочастотных фотонов. Для работы выхода для вещества есть соотношение: $\hbar \omega = \frac{mv^2}{2} + A$

\subsection*{Волновные св-ва эллектронов}
Л. де Бройль обобщил корпускулярно-волновой дуализм до электрона. Согласно ему частице с энергией и импульсом $E, p$ отвечает {\bf монохроматическая волна}, частота и волновой вектор которой завязаны соотношениями $\omega = \frac{E}{\hbar}, {\bf k} = \frac{{\bf p}}{\hbar}$. {\bf Дебройлевская длина волны частицы}: $\lambda = \frac{2\pi \hbar}{ p}$

\subsection*{Волновое уравнение Шредингера.}
Пусть волновое поле описывается { \it скалярной} функцией от времени $t$ и координат ${\bf r} = (x,y,z)$ -- волновой функцией $\psi (t, {\bf r})$. В случае монохроматической волны  $\psi (t, {\bf r}) = A\exp{[-i(\omega t - {\bf k \cdot r})]}$. В нашем курсе только нерелятивистская теория, т.е. энергия частицы массы $m$ равна $E = \frac{{\bf p}^2}{2m}$. Отсюда получаем зависимость частоты дебройлевской волны $\omega$ от волнового вектора ${\bf k}$ ({\bf Закон дисперсии}): $\omega = \frac{\hbar {\bf k}^2}{2m}$.\\
для монохроматической волны имеем: $\frac{\partial \psi}{\partial t} = -i\omega \psi, \nabla \psi = i{\bf k}\psi, \nabla^2\psi = -{\bf k}^2\psi $. Учитывая закон дисперсии, приходим к {\bf Уравнению Шредингера}(УШ) для свободной частицы:\\
\begin{center}
$i\hbar\frac{\partial \psi}{\partial t} = - \frac{\hbar^2}{2m}\nabla^2 \psi$.
\end{center}
Общее решение: $\psi (t, {\bf r})  = \int \dfrac{d^3k}{(2\pi)^{3/2}}C({\bf k})\exp{[-i(\omega t - {\bf k \cdot r})]}$.
\subsubsection*{Аналогия в классической механике}
Для свободной частицы имеем: $H = \frac{{\bf p}}{2m}, S(t,{\bf r}) = \int_0^tLdt, L = {\bf v \cdot p} - H, \nabla S = {\bf p}, \frac{\partial S}{\partial t} = - H$, что дает нам
{\bf уравнение Гамильтона-Якоби} (УГЯ). УГЯ выглядит так:
\begin{center}
$\frac{\partial S}{\partial t} + \frac{1}{2m}(\nabla S)^2 = 0$.
\end{center}
У уравнения есть интеграл $S = -Et+ S(t, {\bf r})$\\
Стационарное УГЯ: $E =  \frac{1}{2m}(\nabla S)^2 $\\
Устанавливаем связь с УШ: $S = -Et + {\bf p \cdot r} = -\hbar (\omega t - {\bf k \cdot r})$\\
Волновая функция выражается через S: $\psi = \exp{[\frac{iS}{\hbar}]}$\\
Подставим в УШ: $\frac{\partial S}{\partial t} + \frac{1}{2m}(\nabla S)^2 - \frac{i\hbar}{2m}\nabla^2S = 0$. Оно отличается от УГЯ только последним слагаемым, которое пропорционально $\hbar$. УШ для $\psi$ переходит в УГЯ для S только при $\hbar \rightarrow 0$


\subsection*{УШ в потенциальном поле U(t,{\bf r}).}
УГЯ: $\frac{\partial S}{\partial t} + \frac{1}{2m}(\nabla S)^2 + U = 0$\\
Квантовое обобщение: $\frac{\partial S}{\partial t} + \frac{1}{2m}(\nabla S)^2 + U - \frac{i\hbar}{2m}\nabla^2S = 0$ - нелинейное уравнение на $\psi$, но для объяснения диффракции и интерференции должно выполняться, что возмущение, создаваемое при прохождении ряда волн, должно быть суммой возмущений отдельных волн. Поэтому нужно линейное уравнение:
\begin{center}
$i\hbar\frac{\partial \psi}{\partial t} = \hat{H}\psi, \hat{H} =  - \frac{\hbar^2}{2m}\nabla^2 + U(t,{\bf r}) - $ оператор Гамильтона.
\end{center}
Получили {\bf уравнение Шредингера для частицы в потенциальном поле}.













































\section {Волновая функция.}
В экспериментах по дифракции пучка электронов проявляются волновые свойства электронов, причем аналогия с дифракцией электромагнитных волн, рассматриваемых как поток фотонов, приводит к статистическому предположению: интенсивность волны в данной точке пространства пропорциональна плотности частиц. Оказывается, что дифракционная картина не зависит от интенсивности пучка частиц: она возникает и при очень малой интенсивности и даже при пропускании одиночных электронов. Таким образом мы пришли к вероятностной интерпретации волновой функции частицы: величина $|\psi(t, {\bf r})|^2$ -- это плотность вероятности обнаружить частицу в точке пространства r в момент времени t. Обозначим $\psi(q, t) = \psi(t, {\bf r})$, где q - совокупность координат квантовой системы. За $dq$ обозначим элемент объема конфигурационного пр-ва.\\
Состояние системы может быть описано определенной комплекснозначной волновой функцией $\psi(q)$. Тогда величина $|\psi|^2dq$ показывает вероятность, что произведенное измерение над системой обнаружит частицу в элементе $dq$ конфигурационного пр-ва. \\
Запишем два УШ для $\psi, \psi^*$:
\begin{center}
$i\hbar\frac{\partial \psi     }{\partial t} = U\psi     - \frac{\hbar^2}{2m}\nabla^2\psi     = 0$\\
$i\hbar\frac{\partial \psi^*}{\partial t} = U\psi^* - \frac{\hbar^2}{2m}\nabla^2\psi^*= 0$, откуда получаем:\\
$\frac{\partial}{\partial t}(\psi\psi*) = -\frac{\hbar}{2mi}\nabla(\psi*\nabla\psi - (\nabla\psi*)\psi)$
\end{center}
Введем плотность и потом вероятности: $\rho, {\bf j}$: $\rho = |\psi|^2, {\bf j} = \frac{\hbar}{2mi}\nabla(\psi*\nabla\psi - (\nabla\psi*)\psi)$. Находим уравнение непрерывности: $\frac{\partial \rho}{\partial t} + \nabla \cdot {\bf j} = 0$. Проинтегрируем по объему V, ограниченному поверхностью $\Sigma$: $\frac{d}{dt}\int_V\rho d^3x = - \oint_{\Sigma}({\bf j \cdot n})d\Sigma$. В предположении $\Sigma \rightarrow \infty; {\bf j}\rightarrow 0$, получим $\int|\psi|^2d^3x = const $. Откуда можно всегда получить условие нормировки: $\int |\psi|^2dq = 1$. Вероятность найти частицу во всем пространстве равна единице, как и должно быть.\\
Заметим, что $\rho, {\bf j}$ инварианты относительно преобразования волновой функции на фазовый множитель $e^{i\alpha}$: $\psi \rightarrow \psi' = e^{i\alpha}\psi; \psi^* \rightarrow \psi'* = e^{i\alpha}\psi^*$. Функции $\psi, \psi'$ отвечают одному состоянию. Пусть $\psi = \sqrt{\rho}e^{i\theta}$. Тогда ${\bf j} = \frac{\hbar}{m}\rho\nabla\theta$. Для частицы с энергией E и импульсом {\bf p} имеем:
\begin{center}
$\psi = A\exp{[-\frac{i}{\hbar}(Et - {\bf p \cdot r})]}, {\bf j} = |A|^2\frac{{\bf p}}{m} = \rho{\bf v}$
\end{center}
\section {Наблюдаемые и операторы.}
Наблюдаемая - физ. величина, значение которой может быть измерено.
Собственные значения физической величины, характеризующей состояние системы $f$: $f_0, f_1, ...$. Их совокупность - спектр собственных значений $f$.
Описание состояния системы осуществляется заданием функции $\psi(q)$.
Собственные волновые функции: $\psi_0, \psi_1, ...$. $\psi_i$ - собственные функции $f$ с соответствующими собственными значениями.\\
Условние нормировки: $\int |\psi_i|^2dq = 1$. Из принципа суперпозиции имеем для произвольного состояния $\psi$:
\begin{itemize}
\item для дискретного спектра:  $\psi = \Sigma_i c_i\psi_i$
\item для недискретного спектра: $\psi = \int c_f\psi_fdf$
\end{itemize}
Таким образом любая волновая функция может быть разложена по собственным функциям физической величины. \{$\psi_i$\} - полная система функций.\\
$|c_i|^2$ определяет вероятность $f_i$ в состоянии $\psi$. Следовательно, $\Sigma |c_i|^2 = 1$.\\
Среднее значение $f$: $\bar{f} = \Sigma |c_i|^2f_i$.\\
Рассмотрим оператор $\hat{f}$: $\bar{f} = \int \psi^*(\hat{f}\psi)dq$, следовательно он линейный. Таким образом мы сопоставили любому наблюдаемому в кв. мех. определенной линейный оператор.
\begin{center}
$\bar{f} = \int \psi_i^*(\hat{f}\psi_i)dq = f_i$, следовательно $\hat{f}\psi_i = f_i\psi_i$.
\end{center}
Собственные значения и средние значения вещественной физической величины в любом состоянии вещественны. Оператор $\hat{f}$ - эрмитов. Также имеем, что $\int \psi_i \psi_j^* = \delta_{ij}$. То есть с.ф. взаимноортогонольны, поэтому \{$\psi_i$\} - полная система ортонормированных функций.\\


\section {Тождественные частицы. Принцип Паули.}
\subsection {Тождественные частицы}
В \textbf{классической механике} даже одинаковые частицы можно различить по положению в пространстве и импульсам. Если частицы в какой-то момент времени пронумеровать, то в следующие моменты времени можно проследить за траекторией любой из них. Классические частицы, таким образом, обладают индивидуальностью, поэтому классическая механика систем из одинаковых частиц принципиально не отличается от классической механики систем из различных частиц. 

В \textbf{квантовой механике} положение иное. Из соотношения неопределенности вытекает, что для микрочастиц вообще неприменимо понятие траектории; состояние микрочастицы описывается волновой функцией, позволяющей лишь вычислять вероятность $|\psi|^2$ нахождения микрочастицы в окрестностях той или иной точки пространства. Если же волновые функции двух тождественных частиц в пространстве перекрываются, то разговор о том, какая частица находится в данной области, вообще лишен смысла: можно говорить лишь о вероятности нахождения в данной области одной из тождественных частиц. Таким образом, в квантовой механике тождественные частицы полностью теряют свою индивидуальность и становятся неразличимыми. Следует подчеркнуть, что принцип неразличимости тождественных частиц не является просто следствием вероятной интерпретации волновой функции, а вводится в квантовую механику как новый принцип, как указывалось выше, является фундаментальным.

Рассмотрим систему из двух одинаковых частиц $\xi_1, \xi_2$. Их волновая функция 

$$
\psi (\xi_1, \xi_2) = e^{i\alpha}\psi (\xi_2, \xi_1) =e^{2i\alpha} \psi (\xi_1, \xi_2), \alpha \in R
$$
Отсюда
$$
e^{2i\alpha} = 1 \Rightarrow e^{i\alpha} = \pm 1 \Rightarrow
$$
$$
\fbox{$\psi (\xi_1, \xi_2) = \pm\psi (\xi_2, \xi_1)$} 
$$
т.е. принцип неразличимости тождественных частиц ведет к определенному свойству симметрии волновой функции. Если при перемене частиц местами волновая функция не меняет знака, то она называется симметричной, если меняет – антисимметричной. Изменение знака волновой функции не означает изменения состояния, т.к. физический смысл имеет лишь квадрат модуля волновой функции.
Волновые функции всех состояний должны иметь одинаковую симметрию, иначе для суперпозиции состояний различной симметрии волновая функция ни симметрична, ни асимметрична.
То есть для произвольного числа одинаковых частиц: \textbf{если какая-либо пара описывает (анти)симметричную волновую функцию, то любая другая пара должна обладать тем же свойством}.\\
Bолновые функции:
\begin{itemize}
	\item симметричные (не меняют знак при перестановке пары частиц) подчиняются статистике Бозе-Энштейна \textbf{(бозоны)}
	\item асимметричные (меняют знак при перестановке пары частиц) подчиняются статистике Ферми-Дирака \textbf{(фермионы)}
\end{itemize}
Из законов релятивистской механики статистика, которой подчиняются частицы, однозначно связана с их спином:
\begin{itemize}
 	\item бозоны $\leftrightarrow$ полуцелый спин
 	\item фермионы $\leftrightarrow$ целый спин
\end{itemize}
Статистика сложных частиц определяется четностью числа входящих в их состав фермионов:
\begin{itemize}
	\item нечетное число фермионов $\leftrightarrow$  Ферми-Дирака
	\item четное число фермионов $\leftrightarrow$  Бозе-Энштейна
\end{itemize}
\subsection {Принцип Паули.}
Рассмотрим систему из N одинаковых частиц, взаимодействием которых можно пренебречь. \\
$\psi_1, ..., \psi_N$ - волновые функции стационарных состояний отдельной частицы. Ищем волновую функцию всей системы.\\
$p_1,...,p_N$ - номера состояний, в которых находятся отдельные частицы.
\\
\textbf{Фермионы:}\\
$\psi$ - антисимметричная комбинация произведений волновых функций.
$$
\fbox{$
\psi_{N_1N_2...} = \frac{1}{\sqrt{N!}}
\begin{vmatrix}
\psi_{p_1}(\xi_1) & ... & \psi_{p_1}(\xi_N)\\
... &...&...\\
\psi_{p_N}(\xi_1) & ... & \psi_{p_N}(\xi_N)
\end{vmatrix}
$}
$$
Перестановка двух частиц эквивалентна перестановке двух столбцов определителя, то есть перемене знака.
Если среди номеров $p_1, ...p_N$ есть два одинаковых, то определитель обращается в 0 (две одинаковые строки).
\\
Отсюда следует \\
\textbf{\underline{Принцип Паули}: в системе одинаковых фермионов не могут одновременно находится в одинаковом состоянии две (и более) частицы.}
\\
\\
(Дальше вроде как можно не писать, потому что это нигде не используется и билет закончился)\\
\textbf{Бозоны :}\\
$$
\fbox{$
	\psi_{N_1N_2...} =\sqrt\frac{N_1!N_2!...}{N!} \sum \psi_{p_1}(\xi_1)...\psi_{p_N}(\xi_N)$}
$$
\begin{itemize}
	\item Суммирование тут ведется по всем перестановкам различных индексов $p_1, ...p_N$
	\item $N_i$ = количество индексов, имеющих одинаковое значение i ($\sum_{i}N_i = N$)
	\item $\sqrt\frac{N_1!N_2!...}{N!}$ - общее число членов в сумме $\sum \psi_{p_1}(\xi_1)...\psi_{p_N}(\xi_N)$ (нормировочный коэффициент для $\int |\psi|^2d\xi_1...d\xi_N = 1$)
\end{itemize}
\end{document}
