%physics2021.tex
\documentclass[specialist, subf, href, colorlinks=true, 12pt, times, mtpro, final]{disser}
\usepackage[utf8]{inputenc}
\usepackage[english,russian]{babel}
\usepackage{a4wide}
\usepackage{mathtext}
\usepackage{amsbsy}
\usepackage{amsthm}
\usepackage{amsmath}
\usepackage{amssymb}
\usepackage{amsfonts}
\usepackage{tikz}
\usepackage{verbatim}
\usepackage{graphicx}
\usepackage{xcolor}
\usepackage{hyperref}
 
 % Цвета для гиперссылок
\definecolor{linkcolor}{HTML}{543AE8} % цвет ссылок
\definecolor{urlcolor}{HTML}{543AE8} % цвет гиперссылок
\hypersetup{pdfstartview=FitH,  linkcolor=linkcolor,urlcolor=urlcolor, colorlinks=true}
 
 % Цвет для комментариев вида \note{...}
\definecolor{faded}{gray}{0.6}
\def\note{\textcolor{faded}}
 
\begin{document}

\tableofcontents

\newpage
\noindent {\Large \bf Программа экзамена по физике (весна 2021)}
\noindent Лектор: {\bf Борисов Анатолий Викторович}\\

\begin{enumerate}
{\footnotesize
\item Уравнение Шрёдингера.
\item Волновая функция.
\item Наблюдаемые и операторы.
\item Принцип суперпозиции.
\item Соотношение неопределённостей Гейзенберга.
\item Изменение наблюдаемых со временем.
\item Гармонический осциллятор.
\item Оператор момента импульса.
\item Спин.
\item Уравнение Паули.
\item Движение в центрально-симметричном поле.
\item Атом водорода.
\item Тождественные частицы. Принцип Паули.
\item Каноническое распределение (распределение Гиббса).
\item Термодинамические параметры и потенциалы.
\item Идеальный классический газ. Распределение Больцмана. 
\item Распределение Гиббса с переменным числом частиц. 
\item Распределение Ферми-Дирака. Идеальный ферми-газ.
\item Распределение Бозе-Эйнштейна. Идеальный бозе-газ.
\item Распределение Планка. Фотонный газ.
}
\end{enumerate}

\section {Уравнение Шрёдингера.}
\note{текст билета}

\section {Волновая функция.}
В экспериментах по дифракции пучка электронов проявляются волновые свойства электронов, причем аналогия с дифракцией электромагнитных волн, рассматриваемых как поток фотонов, приводит к статистическому предположению: интенсивность волны в данной точке пространства пропорциональна плотности частиц. Оказывается, что дифракционная картина не зависит от интенсивности пучка частиц: она возникает и при очень малой интенсивности и даже при пропускании одиночных электронов. Таким образом мы пришли к вероятностной интерпретации волновой функции частицы: величина $|\psi(t, {\bf r})|^2$ -- это плотность вероятности обнаружить частицу в точке пространства r в момент времени t. Обозначим $\psi(q, t) = \psi(t, {\bf r})$, где q - совокупность координат квантовой системы. За $dq$ обозначим элемент объема конфигурационного пр-ва.\\
Состояние системы может быть описано определенной комплекснозначной волновой функцией $\psi(q)$. Тогда величина $|\psi|^2dq$ показывает вероятность, что произведенное измерение над системой обнаружит частицу в элементе $dq$ конфигурационного пр-ва. \\
Запишем два УШ для $\psi, \psi^*$:
\begin{center}
$i\hbar\frac{\partial \psi     }{\partial t} = U\psi     - \frac{\hbar^2}{2m}\nabla^2\psi     = 0$\\
$i\hbar\frac{\partial \psi^*}{\partial t} = U\psi^* - \frac{\hbar^2}{2m}\nabla^2\psi^*= 0$, откуда получаем:\\
$\frac{\partial}{\partial t}(\psi\psi*) = -\frac{\hbar}{2mi}\nabla(\psi*\nabla\psi - (\nabla\psi*)\psi)$
\end{center}
Введем плотность и потом вероятности: $\rho, {\bf j}$: $\rho = |\psi|^2, {\bf j} = \frac{\hbar}{2mi}\nabla(\psi*\nabla\psi - (\nabla\psi*)\psi)$. Находим уравнение непрерывности: $\frac{\partial \rho}{\partial t} + \nabla \cdot {\bf j} = 0$. Проинтегрируем по объему V, ограниченному поверхностью $\Sigma$: $\frac{d}{dt}\int_V\rho d^3x = - \oint_{\Sigma}({\bf j \cdot n})d\Sigma$. В предположении $\Sigma \rightarrow \infty; {\bf j}\rightarrow 0$, получим $\int|\psi|^2d^3x = const $. Откуда можно всегда получить условие нормировки: $\int |\psi|^2dq = 1$. Вероятность найти частицу во всем пространстве равна единице, как и должно быть.\\
Заметим, что $\rho, {\bf j}$ инварианты относительно преобразования волновой функции на фазовый множитель $e^{i\alpha}$: $\psi \rightarrow \psi' = e^{i\alpha}\psi; \psi^* \rightarrow \psi'* = e^{i\alpha}\psi^*$. Функции $\psi, \psi'$ отвечают одному состоянию. Пусть $\psi = \sqrt{\rho}e^{i\theta}$. Тогда ${\bf j} = \frac{\hbar}{m}\rho\nabla\theta$. Для частицы с энергией E и импульсом {\bf p} имеем:
\begin{center}
$\psi = A\exp{[-\frac{i}{\hbar}(Et - {\bf p \cdot r})]}, {\bf j} = |A|^2\frac{{\bf p}}{m} = \rho{\bf v}$
\end{center}
\section {Наблюдаемые и операторы.}
Наблюдаемая - физ. величина, значение которой может быть измерено.
Собственные значения физической величины, характеризующей состояние системы $f$: $f_0, f_1, ...$. Их совокупность - спектр собственных значений $f$.
Описание состояния системы осуществляется заданием функции $\psi(q)$.
Собственные волновые функции: $\psi_0, \psi_1, ...$. $\psi_i$ - собственные функции $f$ с соответствующими собственными значениями.\\
Условние нормировки: $\int |\psi_i|^2dq = 1$. Из принципа суперпозиции имеем для произвольного состояния $\psi$:
\begin{itemize}
\item для дискретного спектра:  $\psi = \Sigma_i c_i\psi_i$
\item для недискретного спектра: $\psi = \int c_f\psi_fdf$
\end{itemize}
Таким образом любая волновая функция может быть разложена по собственным функциям физической величины. \{$\psi_i$\} - полная система функций.\\
$|c_i|^2$ определяет вероятность $f_i$ в состоянии $\psi$. Следовательно, $\Sigma |c_i|^2 = 1$.\\
Среднее значение $f$: $\bar{f} = \Sigma |c_i|^2f_i$.\\
Рассмотрим оператор $\hat{f}$: $\bar{f} = \int \psi^*(\hat{f}\psi)dq$, следовательно он линейный. Таким образом мы сопоставили любому наблюдаемому в кв. мех. определенной линейный оператор.
\begin{center}
$\bar{f} = \int \psi_i^*(\hat{f}\psi_i)dq = f_i$, следовательно $\hat{f}\psi_i = f_i\psi_i$.
\end{center}
Собственные значения и средние значения вещественной физической величины в любом состоянии вещественны. Оператор $\hat{f}$ - эрмитов. Также имеем, что $\int \psi_i \psi_j^* = \delta_{ij}$. То есть с.ф. взаимноортогонольны, поэтому \{$\psi_i$\} - полная система ортонормированных функций.\\

\section{Принцип суперпозиции.}
%{\bf Из Гальцова:}\\
%Здесь и далее используются скобочные обозначения Дирака: вектор состояния (кет-вектор) обозначается скобкой $|\psi\rangle$, сопряженный вектор (бра-вектор) -- $\langle\psi|$, а скалярное призведение $(\psi, \varphi) = \langle\psi|\varphi\rangle$. В этих обозначениях скалярное произведение образуется всякий раз, когда бра-вектор стоит слева от кет-вектора. Норма вектора есть $||\psi|| = \left(\langle\psi|\psi\rangle\right)^{1/2}$\\

Уравнение Шрёдингера линейно, операторы наблюдаемых линейны $=>$ выполняется {\bf принцип суперпозиции:}
\begin{enumerate}
\item Если квантовая система может находиться в состояниях, описываемых волновыми функциями $\psi_1$ и $\psi_2$, то она может находиться и в состоянии $\psi = C_{1}\psi_1 + C_{2}\psi_2$, где $C_1, C_2 \in \mathbb{C}$.
\item $\psi$ и $C\psi,\ C\in\mathbb{C}$ описывают одно и то же состояние.
\end{enumerate}

Таким образом, для физически реализуемых состояний $||\psi|| < \infty$ можно выбрать $||\psi|| = 1$.

\subsection*{Общее решение уравнения Шрёдингера}
Рассмотрим стационарное уравнение Шрёдингера (потенциальная энергия не зависит от $t$)
$$
  \nabla^2\psi + \frac{2m_0}{\hbar^2}\left(E - V(\overline{r})\right)\psi = 0
$$
Когда мы предъяляем требования к $\psi$ (непрерывность, непрерывность производной, однозначность и т.п.), решение существует лишь при определённом значении параметра: $E = E_1, E_2, ...$ -- {\bf энергетические уровни} системы. Решения -- {\bf собственные функции} $\psi_1, \psi_2, ...$
Частные решения имеют вид 
$$
\psi_n (\overline{r}, t) = e^{-\frac{i}{\hbar}E_{n}t}\psi_n(\overline{r}),\ \ \psi^{*}_{n} (\overline{r}, t) = e^{-\frac{i}{\hbar}E_{n}t}\psi^{*}_{n}(\overline{r})
$$
Из линейности уравнения и принципа суперпозиции следует, что общее решение есть линейная комбинация частных:
$$
\psi = \sum_{n}C_n e^{-\frac{i}{\hbar}E_{n}t}\psi_n,\ \ \psi^{*} = \sum_{n}C^{*}_{n} e^{-\frac{i}{\hbar}E_{n}t}\psi^{*}_{n}
$$
Условие нормировки записывается как
$$
\sum_{n, n'}C_{n'}^{*}C_n e^{-\frac{i}{\hbar}t(E_n - E_{n'})}\int\psi_{n'}^{*}\psi_{n}d^{3}x = 1
$$
С условием ортонормированности $\int\psi^{*}_{n'}\psi_{n}d^3x = \delta_{nn'}$ получаем 
$$
\sum_{n}C^{*}_{n}C_n = 1
$$
Тогда интерпретация такова: $|C_n|^2$ характеризует вероятность нахождения частицы в состоянии $n$ (напомним, $|\psi_n|^2$ трактуется как плотность вероятности распределения по пространству частицы, находящейся в состоянии $\psi_n$).

\subsection*{Ансамбли}
\noindent{\bf Квантовые/чистые:} сумма для волновых функций.\\
Совокупность невзаим. друг с другом частиц, которые могут находиться в состоянии $n_1$ или $n_2$:
$$
\psi = C_{n_1}e^{-\frac{i}{\hbar}E_{n_1} t}\psi_{n_1} + C_{n_2}e^{-\frac{i}{\hbar}E_{n_2} t}\psi_{n_2}
$$ 
$$
\psi^{*}\psi = C_{n_1}^{*}C_{n_1}\psi^{*}_{n_1}\psi_{n_1} + C_{n_2}^{*}C_{n_2}\psi^{*}_{n_2}\psi_{n_2} + C_{n_2}^{*}C_{n_1}e^{-\frac{i}{\hbar}t(E_{n_1} - E_{n_2})}\psi_{n_2}^{*}\psi_{n_1} + C_{n_1}^{*}C_{n_2}e^{-\frac{i}{\hbar}t(E_{n_1} - E_{n_2})}\psi_{n_1}^{*}\psi_{n_2}
$$
Смешанные члены с $C_{n_2}^{*}C_{n_1}$ и $C_{n_1}^{*}C_{n_2}$ определяют {\bf статистическую связь} между невзаим. частицами в разных состояниях.\\
Плотность вероятности $|\psi|^2 = |C_1|^2|\psi_1|^2 + |C_2|^2|\psi_2|^2 + 2\text{Re} C_1^{*}C_2\psi^{*}_1\psi_2$\\
Квадрат нормы $(\psi,\psi) = 1 = |C_1|^2 + |C_2|^2 + 2\text{Re} C_1^{*}C_2(\psi_1,\psi_2)$\\
Среднее значение наблюдаемой $\langle A\rangle = (\psi, \hat{A}\psi) = |C_1|^2(\psi_1,\hat{A}\psi_1) + |C_2|^2(\psi_2,\hat{A}\psi_2) + 2\text{Re} C_1^{*}C_2(\psi_2,\hat{A}\psi_2)$\\
Отсюда видно, что квантовая механика не сводится к классической теории вероятности: возникает характерный эффект интерференции состояний $\psi_1$ и $\psi_2$, не имеющий классического аналога.\\

\noindent{\bf Смешанные:} суммая для вероятностей $|C|^2 = |C_1|^2 + |C_2|^2$.\\
Нет статистической связи между различными состояниями и типичных волновых процессов (классическая теория).\\
В волновых процессах члены $C_2^{*}C_1$ и $C_1^{*}C_2$ могут исчезать когда фаза между различными квантовыми состояниями быстро изменяется (некогерентный свет).

\subsection*{Разложение состояний}
Состояния мы задаём как функционалы $\langle\omega|A\rangle$ со свойствами:
\begin{enumerate}
	\item $\langle\lambda A + B|\omega\rangle = \lambda\langle A|\omega\rangle + \langle B|\omega\rangle$
	\item $\langle A^2|\omega\rangle \ge 0$ (среднее значение $A^2$ неотрицательно)
	\item $\langle C|\omega\rangle = C$ (среднее значение $C$ совпадает с константой для любого состояния)
	\item $\langle A|\omega\rangle = \langle A|\omega\rangle$ (средние значения вещественны)
\end{enumerate}
Такие функционалы можно задать следующим образом:
$$
	\langle A|\omega\rangle = \text{Tr}MA,
$$
$M$ -- оператор в $\mathbb{C}^n$ такой, что $M^* = M,\ (M\xi,\xi)\ge 0,\ \text{Tr}M = 1$.\\
$M$ -- {\bf матрица плотности}.\\
Выпуклая комбинация $M = \alpha M_1 + (1-\alpha)M_2,\ 0<\alpha<1$, обладает теми же свойствами, что и соответствующее состояние $\omega = \alpha\omega_1 + (1-\alpha)\omega_2$.\\
Состояние, не раскладывающееся в выпуклую комбинацию других -- {\bf чистое}. Любое состояние является выпуклой комбинацией чистых состояний.
\section{Соотношение неопределённостей Гейзенберга.}
\noindent{\bf Утверждение.} Несколько наблюдаемых $(F_1, F_2, ...)$ одновременно измеримы (имеют с достоверностью определённые значения для собственных состояний $|f\rangle$) тогда и только тогда, когда $[F_i,F_j] = 0,\ \forall i, j$. \note{(коммутатор равен нулю)}
\begin{proof}
Рассмотрим случай двух наблюдаемых.\\
Необходимость:
$$
\begin{aligned}
&F_1|f_1,f_2,...\rangle = f_1|f_1,f_2,...\rangle\\
&F_2|f_1,f_2,...\rangle = f_2|f_1,f_2,...\rangle\\
&......
\end{aligned}
\eqno(1)
$$
Необходимость очевидна из (1), если к первому соотношению оператор $F_2$, ко второму $F_1$ и произвести вычитание, то получим желаемое.\\
Достаточность:\\
Предположим, что в спектре оператора $F_1$ каждому собственному значению отвечает единственный собственный вектор (т.е. спектр простой).
$$
F_2 (F_1 |f_1,f_2\rangle) = f_1 F_2 |f_1,f_2\rangle = F_1 (F_2 |f_1,f_2\rangle)
$$
$=>\ F_2|f_1,f_2\rangle$ -- с.в. оператора $F_1$ с с.з. $f_1$.\\
В силу простоты спектра $F_1$, $F_2|f_1,f_2\rangle = \text{const} |f_1, f_2\rangle$.
Полагая $\text{const} = f_2$, получим, что вектор также собственный для $F_2$.\\
Поскольку система с.в. $F_1$ полна, то система общих с.в. $F_1$ и $F_2$ полна. В случае, если среди с.з. $F_1$ есть совпадающие, приведённое построение необходимо дополнить построением определённой линейной комбинации векторов, принадлежащих одному с.з.
\end{proof}
Рассмотрим пару некоммутирующих наблюдаемых: самосопряжённых операторов $A^+ = A,\ B^+ = B$. Тогда
$$
[A, B] = iC,
$$
где $C^+ = C$. Определим {\bf дисперсии} величин равенствами
$$
(\delta A)^2 = \langle(A-\langle A\rangle)^2\rangle,\ \ (\delta B)^2 = \langle(B-\langle B\rangle)^2\rangle,\ \ (\delta C)^2 = \langle(C-\langle C\rangle)^2\rangle,
$$
где усреднение проводится по выбранному состоянию $|\psi\rangle$. Тогда
$$
(\delta A)^2(\delta B)^2 \ge \frac{1}{4}\langle C\rangle^2
$$
Это ограничение называется {\bf соотношением неопределенностей} (СН) и получено впервые Гейзенбергом в 1927 г. для частного случая
наблюдаемых $x$ (координата) и $p_x$ (импульс):
$$
[x,p_x] = i\hbar I\ =>\ (\delta x)^2(\delta p_x)^2 \ge \frac{\hbar^2}{4}
$$
Это ограничение называют {\bf соотношением неопределённостей Гейзенберга}.\\
Для коммутирующих наблюдаемых правая часть СН обращается в нуль, что соответствует одновременной измеримости таких наблюдаемых.
Для некоммутирующих наблюдаемых СН накладывает ограничение на точности, с которыми могут быть одновременно заданы (измерены) эти наблюдаемые.\\
Найдем состояния $\psi$, в которых достигается минимум неопределенностей, т. е. точное равенство в СН. Получаем для них систему уравнений:
$$
\begin{aligned}
&(\lambda a - ib)\psi = 0,\\
&\lambda^2\langle a^2\rangle + \lambda \langle C\rangle + \langle b^2\rangle = 0,\\
&\langle a^2\rangle\langle b^2\rangle = \frac{1}{4}\langle C\rangle^2.
\end{aligned}
$$
Отсюда находим
$$
\lambda = -\frac{\langle C\rangle}{2\langle a^2\rangle}
$$
и уравнение для определения {\bf состояния, минимизирующего произведение неопределённостей}, принимает вид
$$
\left(\frac{\langle C\rangle}{2\langle a^2\rangle}a + ib\right)\psi = 0.
$$
\section{Изменение наблюдаемых со временем.}

\section {Уравнение Паули.}
\subsection {Тождественные частицы}
Спин в аппарат квантовой механики был введен Паули. Он
предложил \textbf{постулировал} для описания электрона уравнение,
которое теперь называется \textbf{уравнением Паули}:


$$ i\hbar\frac{\partial\psi}{\partial t}=\hat{H}_P\psi$$
$$ \hat{H}_P = \frac{1}{m_e} \left( \hat{p} - \frac{e}{c}A \right) + e\Phi - \hat \mu \hat B$$, 

Паулиевский гамильтониан $\hat{H}_P $ отличается от шрёдингеровского
добавлением слагаемого $\hat U_p = - \hat \mu \hat B$
, описывающего взаимодействие с магнитным полем
$B = \nabla \times A$.
спинового магнитного момента
электрона, представляемого оператором.
$\hat \mu$ - собственный магнитный момент частицы.

Рассмотрим «вывод» уравнения Паули, принадлежащий
Фейнману (R. P. Feynman). Из основного соотношения для матриц
Паули:

$$\sigma_k \sigma_n = \delta_{kn} + i \epsilon_{kns} \sigma_{s},$$

Следует

$$(\sigma \cdot a) (\sigma \cdot b) = a \cdot b + i \sigma \cdot (a \times b),$$

где $a$, $b$ - произвольные вектора. Учитывая это запишем гамильтониан электрона в электростатическом поле $U = e \Phi$ в эквивалетном шрёдингеровскому виде


$$ \hat{H} = \frac{(\sigma \cdot \hat p)^2} {2m_e} + e\Phi$$

Введем взаимодействие с магнитным полем (это постулат)

$$\hat p \rightarrow \hat P = \hat p  - \frac{e}{c} A$$

Тогда получим гамильтониан


$$ \hat{H}_F = \frac{(\sigma \cdot \hat P)^2} {2m_e} + e\Phi, $$

который эквивалентен гамильтониану Паули $\hat H_F = \hat H_P$. Действительно, 

$$ (\sigma \cdot \hat P)^2 = \hat P^2 + i \sigma \cdot (\hat P \times \hat P),$$

где второе слагаемое отлично от нуля ввиду некоммутативности
компонент оператора кинетического импульса $\hat p$:

$$\left[ \hat P_n, \hat P_k \right] = [\hat p_n - \frac{e}{c}A_n, \hat p_k - \frac{e}{c}A_k] = i \hbar \frac{e}{c}(\partial_n A_k - \partial_k A_n)$$

Следовательно,

$$(\hat P \times \hat P)_s = \epsilon_{snk}\hat P_n \hat P_k = \frac{1}{2} \epsilon_{snk} [\hat P_n, \hat P_k] = i \hbar \frac{e}{c} \epsilon_{snk} \partial_n A_k$$ 

В результате получаем 
$$\frac{(\sigma \cdot \hat P)^2}{2m_e} = \frac{\hat P^2}{2m_e} - \frac{e \hbar}{2m_e c} \sigma \cdot B,$$

т. е. приходим к паулиевскому взаимодействию спинового
магнитного момента электрона с магнитным полем.

\section {Тождественные частицы. Принцип Паули.}
\subsection {Тождественные частицы}
В \textbf{классической механике} даже одинаковые частицы можно различить по положению в пространстве и импульсам. Если частицы в какой-то момент времени пронумеровать, то в следующие моменты времени можно проследить за траекторией любой из них. Классические частицы, таким образом, обладают индивидуальностью, поэтому классическая механика систем из одинаковых частиц принципиально не отличается от классической механики систем из различных частиц. 

В \textbf{квантовой механике} положение иное. Из соотношения неопределенности вытекает, что для микрочастиц вообще неприменимо понятие траектории; состояние микрочастицы описывается волновой функцией, позволяющей лишь вычислять вероятность $|\psi|^2$ нахождения микрочастицы в окрестностях той или иной точки пространства. Если же волновые функции двух тождественных частиц в пространстве перекрываются, то разговор о том, какая частица находится в данной области, вообще лишен смысла: можно говорить лишь о вероятности нахождения в данной области одной из тождественных частиц. Таким образом, в квантовой механике тождественные частицы полностью теряют свою индивидуальность и становятся неразличимыми. Следует подчеркнуть, что принцип неразличимости тождественных частиц не является просто следствием вероятной интерпретации волновой функции, а вводится в квантовую механику как новый принцип, как указывалось выше, является фундаментальным.

Рассмотрим систему из двух одинаковых частиц $\xi_1, \xi_2$. Их волновая функция 

$$
\psi (\xi_1, \xi_2) = e^{i\alpha}\psi (\xi_2, \xi_1) =e^{2i\alpha} \psi (\xi_1, \xi_2), \alpha \in R
$$
Отсюда
$$
e^{2i\alpha} = 1 \Rightarrow e^{i\alpha} = \pm 1 \Rightarrow
$$
$$
\fbox{$\psi (\xi_1, \xi_2) = \pm\psi (\xi_2, \xi_1)$} 
$$
т.е. принцип неразличимости тождественных частиц ведет к определенному свойству симметрии волновой функции. Если при перемене частиц местами волновая функция не меняет знака, то она называется симметричной, если меняет – антисимметричной. Изменение знака волновой функции не означает изменения состояния, т.к. физический смысл имеет лишь квадрат модуля волновой функции.
Волновые функции всех состояний должны иметь одинаковую симметрию, иначе для суперпозиции состояний различной симметрии волновая функция ни симметрична, ни асимметрична.
То есть для произвольного числа одинаковых частиц: \textbf{если какая-либо пара описывает (анти)симметричную волновую функцию, то любая другая пара должна обладать тем же свойством}.\\
Bолновые функции:
\begin{itemize}
	\item симметричные (не меняют знак при перестановке пары частиц) подчиняются статистике Бозе-Энштейна \textbf{(бозоны)}
	\item асимметричные (меняют знак при перестановке пары частиц) подчиняются статистике Ферми-Дирака \textbf{(фермионы)}
\end{itemize}
Из законов релятивистской механики статистика, которой подчиняются частицы, однозначно связана с их спином:
\begin{itemize}
 	\item бозоны $\leftrightarrow$ полуцелый спин
 	\item фермионы $\leftrightarrow$ целый спин
\end{itemize}
Статистика сложных частиц определяется четностью числа входящих в их состав фермионов:
\begin{itemize}
	\item нечетное число фермионов $\leftrightarrow$  Ферми-Дирака
	\item четное число фермионов $\leftrightarrow$  Бозе-Энштейна
\end{itemize}
\subsection {Принцип Паули.}
Рассмотрим систему из N одинаковых частиц, взаимодействием которых можно пренебречь. \\
$\psi_1, ..., \psi_N$ - волновые функции стационарных состояний отдельной частицы. Ищем волновую функцию всей системы.\\
$p_1,...,p_N$ - номера состояний, в которых находятся отдельные частицы.
\\
\textbf{Фермионы:}\\
$\psi$ - антисимметричная комбинация произведений волновых функций.
$$
\fbox{$
\psi_{N_1N_2...} = \frac{1}{\sqrt{N!}}
\begin{vmatrix}
\psi_{p_1}(\xi_1) & ... & \psi_{p_1}(\xi_N)\\
... &...&...\\
\psi_{p_N}(\xi_1) & ... & \psi_{p_N}(\xi_N)
\end{vmatrix}
$}
$$
Перестановка двух частиц эквивалентна перестановке двух столбцов определителя, то есть перемене знака.
Если среди номеров $p_1, ...p_N$ есть два одинаковых, то определитель обращается в 0 (две одинаковые строки).
\\
Отсюда следует \\
\textbf{\underline{Принцип Паули}: в системе одинаковых фермионов не могут одновременно находится в одинаковом состоянии две (и более) частицы.}
\\
\\
(Дальше вроде как можно не писать, потому что это нигде не используется и билет закончился)\\
\textbf{Бозоны :}\\
$$
\fbox{$
	\psi_{N_1N_2...} =\sqrt\frac{N_1!N_2!...}{N!} \sum \psi_{p_1}(\xi_1)...\psi_{p_N}(\xi_N)$}
$$
\begin{itemize}
	\item Суммирование тут ведется по всем перестановкам различных индексов $p_1, ...p_N$
	\item $N_i$ = количество индексов, имеющих одинаковое значение i ($\sum_{i}N_i = N$)
	\item $\sqrt\frac{N_1!N_2!...}{N!}$ - общее число членов в сумме $\sum \psi_{p_1}(\xi_1)...\psi_{p_N}(\xi_N)$ (нормировочный коэффициент для $\int |\psi|^2d\xi_1...d\xi_N = 1$)
\end{itemize}
\section {Каноническое распределение (распределение Гиббса).}
Рассмотрим макроскопическое тело, являющееся малой частью большой замкнутой системы. Будем рассматривать две части: тело и "всё остальное" (среду).\\
\textbf{Микроканоническое распределение} (вероятность нахождения системы в любом из $d\Gamma$ состояний):
$$
d\omega = const \cdot \delta (E + E' - E^{(0)})d\Gamma	d\Gamma'
$$
\begin{itemize}
	\item $d\Gamma$ - число квантовых состояний замкнутой системы, приходящихся на определенный бесконечно малый интервал значений её энергии
	\item $E, d\Gamma$ и $ E', d\Gamma'$ относятся соответственно к телу и среде
	\item $E^{(0)}$ - заданное значение энергии замкнутой системы; этому
	значению должна быть равна сумма $E + E'$ энергий тела и среды
\end{itemize}
Нашей целью является нахождение вероятности $\omega_n$ такого
состояния всей системы, при котором данное тело находится в
некотором определенном квантовом состоянии (с энергией $E_n$),
т. е. в состоянии, описанном микроскопическим образом. Ми­кроскопическим же состоянием среды мы при этом не инте­ресуемся, т. е. будем считать, что она находится в некотором макроскопически описанном состоянии.\\
Пусть $\Delta\Gamma'$ есть стати­стический вес макроскопического состояния среды; обозначим также через $\Delta E'$ интервал значений энергии среды, соответствующий интервалу $\Delta\Gamma'$ квантовых состояний.\\
Искомую вероятность $\omega_n$ мы найдем, заменив в предыдущем уравнении $d\Gamma$ еди­ницей, положив $E = E_n$ и проинтегрировав по $d\Gamma'$:
$$
\omega_n = const \cdot \int \delta(E_n + E' - E^{(0)})d\Gamma'
$$
Пусть $\Gamma'(E')$ - полное число квантовых состояний среды с энергией, меньшей или равной $E'$.\\
Поскольку подынтегральное выражение зависит только от $E'$, можно перейти к интегри­рованию по $dE'$, написав:
$$
d\Gamma' = \frac{d\Gamma'(E')}{dE'}dE'
$$
$$
\frac{d\Gamma'}{dE'} = \frac{e^{S'(E')}}{\Delta E'}
$$
где $S'(E')$ - энтропия среды как функция её энергии.\\
Тогда
$$
\omega_n = const \cdot \int \frac{e^{S'}}{\Delta E'}\delta(E_n + E' - E^{(0)})dE'
$$
Благодаря наличию $\delta$-функции интегрирование сводится к замене $E'$ на $E^{(0)} - E_n$.
Получаем
$$
\omega_n = const \cdot (\frac{e^{S'}}{\Delta E'})_{E' = E^{(0)} - E_n}
$$
Учтем теперь, что вследствие малости тела его энергия $E_n$ мала по сравнению с $E^{(0)}$. Величина $\Delta Е'$ относительно очень
мало меняется при незначительном изменении $E'$; поэтому в ней
можно просто положить $E' = E^{(0)}$, после чего она превратит­
ся в не зависящую от $E_n$ постоянную.\\
В экспоненциальном же множителе $e^{S'}$ надо разложить $S'(E^{(0)} - E_n)$ по степеням $E_n$ сохранив также и линейный член:
$$
S'(E^{(0)} - E_n) = S'(E^{(0)}) - E_n \frac{dS'	(E^{(0)})}{dE^{(0)}}
$$
Но производная от энтропии $S'$ о энергии есть не что иное,
как $1/T$, где $T$ - температура системы (температура тела и сре­ды одинакова, так как система предполагается находящейся в
равновесии).\\
Таким образом, получаем окончательно для $w_n$ следующее
выражение (\textbf{распределение Гибса} или \textbf{каноническое распределение}): 
$$
\fbox{$\omega_n = A e^{-\frac{E_n}{T}}$}
$$
где А - не зависящая от $E_n$ нормировочная постоянная.\\
Это — одна из важнейших формул статистики; она \textbf{определяет ста­тистическое распределение любого макроскопического тела,являющегося сравнительно малой частью некоторой большой замкнутой системы}.\\
Нормировочная постоянная А определяется условием $\sum \omega_n = 1$, откуда
$$
\frac{1}{A} = \sum_{n} e ^{-\frac{E_n}{T}}
$$
Среднее значение любой физической величины f , характеризую­щей данное тело, может быть вычислено с помощью распреде­ления Гиббса по формуле:
$$
f = \sum_{n} \omega_n f_{nn} = \frac{\sum_{n}f_{nn} e^{-\frac{E_n}{T}}}{\sum_{n} e^{-\frac{E_n}{T}}}
$$



 
\end{document}
